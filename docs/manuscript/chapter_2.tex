\chapter{Background concepts}

\label{background}

\section{Explainable artificial intelligence}

In this section, we lay out background concepts for Explainable Artificial Intelligence (XAI) which have been largely adopted from \citet{arrieta2020explainable}. The study is particularly helpful for us since it consists of a vast literature review and survey of recent XAI techniques; as well as discussions into the future of XAI research. We start off by providing the following three definitions from the study:

\begin{definition}[Transparency; \citealt{arrieta2020explainable}]
  A model is considered to be transparent if by itself it is understandable. Since a model can feature different degrees of understandability, transparent models \textelp{} are divided into three categories: simulatable models, decomposable models and algorithmically transparent models. 
\end{definition}

\begin{definition}[Explainability; \citealt{arrieta2020explainable}]
  Explainability is associated with the notion of explanation as an interface between humans and a decision maker that is, at the same time, both an accurate proxy of the decision maker and comprehensible to humans. 
\end{definition}

\begin{definition}[Explainable Artificial Intelligence; \citealt{arrieta2020explainable}]
  Given an audience, an \textbf{explainable} Artificial Intelligence is one that produces details or reasons to make its functioning clear or easy to understand.
\end{definition}

\citet{arrieta2020explainable} notably concretize the presence of a target audience for XAI, implying that different XAI techniques should be employed for different target audiences. In their study, they provide examples of various target audiences such as domain experts, end-users and managers.

\section{Weighted finite-state automaton}

\begin{definition}[Semiring; \citealt{kuich1986linear}]
  A semiring is a set $\mathbb{K}$ along with two binary associative operations $\oplus$ (addition) and $\otimes$ (multiplication) and two identity elements: $\bar{0}$ for addition and $\bar{1}$ for multiplication. Semirings require that addition is commutative, multiplication distributes over addition, and that multiplication by $\bar{0}$ annihilates, i.e., $\bar{0} \otimes a = a \otimes \bar{0} = \bar{0}$.

\begin{remark}
  Semirings follow the following generic notation: $\langle \mathbb{K}, \oplus, \otimes, \bar{0}, \bar{1} \rangle$.
\end{remark}

\begin{remark}
  A simple and common semiring is the real or sum-product semiring: $\langle \mathbb{R}, +, \times, 0, 1 \rangle$. Two important semirings for this thesis are shown below.
\end{remark}

\begin{remark}
  \textbf{Max-sum} semiring: $\langle \mathbb{R} \cup \{-\infty\}, \text{max}, +, -\infty, 0 \rangle$
\end{remark}

\begin{remark}
  \textbf{Max-product} semiring: $\langle \mathbb{R}_{>0} \cup \{-\infty\}, \text{max}, \times, -\infty, 1 \rangle$
\end{remark}

\end{definition}

\begin{definition}[Weighted finite-state automaton; \citealt{peng2018rational}]
  A weighted finite-state automaton over a semiring $\mathbb{K}$ is a 5-tuple $\mathcal{A} = \langle \Sigma, \mathcal{Q}, \mathcal{T}, \lambda, \rho \rangle$, with:

  \begin{itemize}
    \itemsep0em 
    \item[--] a finite input alphabet $\Sigma$;
    \item[--] a finite state set $\mathcal{Q}$;
    \item[--] transition weights $\mathcal{T}: \mathcal{Q} \times \mathcal{Q} \times (\Sigma \cup \{\epsilon\}) \rightarrow \mathbb{K}$;
    \item[--] initial weights $\lambda: \mathcal{Q} \rightarrow \mathbb{K}$; 
    \item[--] and final weights $\rho: \mathcal{Q} \rightarrow \mathbb{K}$.
  \end{itemize}

  \begin{remark}
    $\epsilon \notin \Sigma$ refers to special $\epsilon$-transitions that may be taken without consuming any input.
  \end{remark}

  \begin{remark}
    Self-loop transitions in $\mathcal{A}$ refer to special transitions which consume an input while staying at the same state.
  \end{remark}
  
  \begin{remark}
    $\Sigma^{*}$ refers to the (possibly infinite) set of all strings over the alphabet $\Sigma$.
  \end{remark}
   
\end{definition}

\begin{definition}[Path score; \citealt{peng2018rational}]

  Let $\pmb{\pi} = \langle \pi_1, \pi_2, \dots, \pi_n \rangle$ be a sequence of adjacent transitions in $\mathcal{A}$, with each $\pi_i = \langle q_i, q_{i+1}, z_i \rangle \in \mathcal{Q} \times \mathcal{Q} \times (\Sigma \cup \{\epsilon\})$. The path $\pmb{\pi}$ derives the $\epsilon$-free string $\pmb{x} = \langle x_1, x_2, \dots, x_m \rangle \in \Sigma^{*}$; which is a substring of the $\epsilon$-containing string $\pmb{z} = \langle z_1, z_2, \dots, z_n \rangle \in (\Sigma \cup \{\epsilon\})^{*}$. $\pmb{\pi}$'s score in $\mathcal{A}$ is given by:
  
\end{definition}

\begin{equation}
  \mathcal{A}[\pmb{\pi}] = \lambda(q_1) \otimes \Bigg( \bigotimes_{i=1}^n \mathcal{T}(\pi_i) \Bigg) \otimes \rho(q_{n+1})
\end{equation}

\begin{definition}[String score; \citealt{peng2018rational}]

Let $\Pi(\pmb{x})$ denote the set of all paths in $\mathcal{A}$ that derive $\pmb{x}$. Then the string score assigned by $\mathcal{A}$ to string $\pmb{x}$ is given by:
  
\end{definition}

\begin{equation}
  \mathcal{A}[\![\pmb{x}]\!] = \bigoplus_{\pmb{\pi} \in \Pi(\pmb{x})} \mathcal{A}[\pmb{\pi}]
\end{equation}

\begin{remark}
  Since $\mathbb{K}$ is a semiring, $\mathcal{A}[\![\pmb{x}]\!]$ can be efficiently computed using the Forward algorithm \citep{baum1966statistical}. Its dynamic program is summarized below without $\epsilon$-transitions for simplicity. $\Omega_i(q)$ gives the aggregate score of all paths that derive the substring $\langle x_1, x_2, \dots, x_i \rangle$ and end in state $q$:
 
\begin{subequations}
  \begin{align}
    \Omega_0(q) &= \lambda(q) \\
    \Omega_{i+1}(q) &= \bigoplus_{q' \in \mathcal{Q}} \Omega_i(q') \otimes \mathcal{T}(q',q,x_i)  \\
    \mathcal{A}[\![\pmb{x}]\!] &= \bigoplus_{q \in \mathcal{Q}} \Omega_n(q) \otimes \rho(q)
  \end{align}
\end{subequations}

\end{remark}

\begin{remark}
  The Forward algorithm can be generalized to any semiring \citep{eisner2002parameter} and has a runtime of $O(|Q|^3 + |Q|^2|\pmb{x}|)$ \citep{schwartz2018sopa}; notably with a linear runtime with respect to the length of the input string $\pmb{x}$.
\end{remark}

\begin{remark}
  A special case of Forward is the Viterbi algorithm, where the addition $\oplus$ operator is contrained to the maximum function \citep{viterbi1967error}. Viterbi therefore returns the highest scoring path $\pmb{\pi}$ that derives the input string $\pmb{x}$.
\end{remark}

%%% Local Variables: 
%%% mode: latex
%%% TeX-master: "main"
%%% End: 
