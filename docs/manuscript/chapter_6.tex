\chapter{Conclusions}

In this thesis, we emphasized the importance of XAI research and laid out clear
definitions of XAI-related concepts. We drew inspiration from the SoPa model in
\citet{schwartz2018sopa} and worked on improving the SoPa model to create our
SoPa++ model. We focused on targetted changed to the SoPa model in order to
create a SoPa++ model that allows for the ability to conduct the post-hoc
explanations by simplification explainability method; where we ultimately
simplify a SoPa++ model into a RE proxy model. This ability can be largely
attributed to the special features of WFAs as well as the flexibility of the
TauSTE layer. With all of these features, we were able to answer our research
questions.

Firstly, we were able to show that SoPa++'s performance range of 96.9-98.3$\%$
falls into the competitive accuracy range of 96.6-99-5$\%$ on the FMTOD English
language intent detection task while comparing with other recent studies. In
this respect, we conclude that SoPa++ offers competitive performance on the
aforementioned data set and task.

Next, we address to what extent SoPa++ allows for effective explanations by
simplification. In order to address this question, we analyzed a variety of
performance scores and distance metrics for SoPa++ and RE proxy models. We were
able to show that larger models with higher $\tau$-thresholds tend to have
highly similar performance scores within $\sim$1-2$\%$ accuracy and mean softmax
distance norms as low as $\sim$4-5$\%$. As a result, we show that explanations
by simplification are effective especially under the aforementioned special
conditions where SoPa++ models are large with high $\tau$-thresholds.

Finally, we address our third research question by probing for interesting and
relevant explanations for classification on the FMTOD English language intent
detection data set. We analyze light variants of SoPa++ and RE proxy model pairs
by looking at their linear weight distributions in TauSTE neurons. Based on this
distribution, we analyze captured activating REs in the RE lookup layer
corresponding to these neurons and found interesting captured REs. These REs
capture semantic and syntactic information which reflect important phrases for
reaching classification decisions. Furthermore, we were also able to identify
some inductive biases in the REs captured, such as those related to gender.

%%% Local Variables: 
%%% mode: latex
%%% TeX-master: "main"
%%% End: 