%%%%%%%%%%%%%%%%%%%%%%%%%%%%%%%%%%%%%%%%%
% Masters/Doctoral Thesis 
% LaTeX Template
% Version 2.5 (27/8/17)
%
% This template was downloaded from:
% http://www.LaTeXTemplates.com
%
% Version 2.x major modifications by:
% Vel (vel@latextemplates.com)
%
% This template is based on a template by:
% Steve Gunn (http://users.ecs.soton.ac.uk/srg/softwaretools/document/templates/)
% Sunil Patel (http://www.sunilpatel.co.uk/thesis-template/)
%
% Template license:
% CC BY-NC-SA 3.0 (http://creativecommons.org/licenses/by-nc-sa/3.0/)
%
%%%%%%%%%%%%%%%%%%%%%%%%%%%%%%%%%%%%%%%%%

%----------------------------------------------------------------------------------------
%	PACKAGES AND OTHER DOCUMENT CONFIGURATIONS
%----------------------------------------------------------------------------------------

\documentclass[
11pt, % The default document font size, options: 10pt, 11pt, 12pt
oneside, % Two side (alternating margins) for binding by default, uncomment to switch to one side
english, % ngerman for German
singlespacing, % Single line spacing, alternatives: onehalfspacing or doublespacing
%draft, % Uncomment to enable draft mode (no pictures, no links, overfull hboxes indicated)
% nolistspacing, % If the document is onehalfspacing or doublespacing, uncomment this to set spacing in lists to single
%liststotoc, % Uncomment to add the list of figures/tables/etc to the table of contents
%toctotoc, % Uncomment to add the main table of contents to the table of contents
%parskip, % Uncomment to add space between paragraphs
%nohyperref, % Uncomment to not load the hyperref package
headsepline, % Uncomment to get a line under the header
%chapterinoneline, % Uncomment to place the chapter title next to the number on one line
% consistentlayout, % Uncomment to change the layout of the declaration, abstract and acknowledgements pages to match the default layout
]{thesis} % The class file specifying the document structure

\usepackage[utf8]{inputenc} % Required for inputting international characters
\usepackage[T1]{fontenc} % Output font encoding for international characters
\usepackage{mathpazo} % Use the Palatino font by default
\usepackage[backend=bibtex,style=authoryear,natbib=true,maxbibnames=99]{biblatex} % Use the bibtex backend with the authoryear citation style (which resembles APA)
\usepackage[autostyle=true]{csquotes} % Required to generate language-dependent quotes in the bibliography
\usepackage{amsthm}
\usepackage{amsmath}
\usepackage{chngcntr}
\usepackage{threeparttable}
\usepackage{ragged2e}
\usepackage{algorithm} 
\usepackage{algpseudocode}
\usepackage{bm}
\usepackage{etoolbox}
\usepackage{acro}
\usepackage{enumitem}

% specialized settings
\addto\captionsenglish{% Replace "english" with the language you use
  \renewcommand{\contentsname}%
  {Table of Contents}
}
\acsetup{list-style = tabular}
\counterwithout{equation}{chapter}
\counterwithout{figure}{chapter}
\counterwithout{table}{chapter}
\theoremstyle{definition}
\newtheorem{definition}{Definition}
\theoremstyle{remark}
\newtheorem{remark}{Remark}[definition]
\addbibresource{../bibtex.bib} % The filename of the bibliography
\graphicspath{{../visuals/}}
\newcolumntype{L}[1]{>{\RaggedRight\hspace{0pt}}p{#1}}
\newcolumntype{R}[1]{>{\RaggedLeft\hspace{0pt}}p{#1}}
\renewcommand{\algorithmicrequire}{\textbf{Input(s):}}
\renewcommand{\algorithmicensure}{\textbf{Output(s):}}
\DeclareMathOperator*{\argmax}{arg\,max}
\algdef{SE}{Begin}{End}{\textbf{begin}}{\textbf{end}}
\algnewcommand\algorithmicforeach{\textbf{for each}}
\algdef{S}[FOR]{ForEach}[1]{\algorithmicforeach\ #1\ \algorithmicdo}

% macro for algorithmic comment
\makeatletter
\AfterEndEnvironment{algorithm}{\let\@algcomment\relax}
\AtEndEnvironment{algorithm}{\kern2pt\hrule\relax\vskip3pt\@algcomment}
\let\@algcomment\relax
\newcommand\algcomment[1]{\def\@algcomment{\footnotesize#1}}
\renewcommand\fs@ruled{\def\@fs@cfont{\bfseries}\let\@fs@capt\floatc@ruled
  \def\@fs@pre{\hrule height.8pt depth0pt \kern2pt}%
  \def\@fs@post{}%
  \def\@fs@mid{\kern2pt\hrule\kern2pt}%
  \let\@fs@iftopcapt\iftrue}
\makeatother

% add link for citeyear
\DeclareCiteCommand{\citeyear}
{}
{\bibhyperref{\printfield{year}}}
{\multicitedelim}
{}


% list all acronyms
\DeclareAcronym{re}{
  short = RE,
  long  = Regular Expression,
  class = abbrev
}
\DeclareAcronym{wfa}{
  short = WFA,
  long  = Weighted Finite-state Automaton,
  class = abbrev
}
\DeclareAcronym{wfas}{
  short = WFAs,
  long  = Weighted Finite-state Automata,
  class = abbrev
}
\DeclareAcronym{fmtod}{
  short = FMTOD,
  long  = Facebook Multilingual Task Oriented Dialog,
  class = abbrev
}
\DeclareAcronym{xai}{
  short = XAI,
  long  = Explainable Artificial Intelligence,
  class = abbrev
}
\DeclareAcronym{ste}{
  short = STE,
  long  = Straight-Through Estimator,
  class = abbrev
}
\DeclareAcronym{tauste}{
  short = TauSTE,
  long  = Tau-threshold Straight-Through Estimator,
  class = abbrev
}
\DeclareAcronym{ml}{
  short = ML,
  long  = Machine Learning,
  class = abbrev
}
\DeclareAcronym{nlp}{
  short = NLP,
  long  = Natural Language Processing,
  class = abbrev
}
\DeclareAcronym{nfa}{
  short = NFA,
  long  = Nondeterministic Finite-state Automaton,
  class = abbrev
}
\DeclareAcronym{nfas}{
  short = NFAs,
  long  = Nondeterministic Finite-state Automata,
  class = abbrev
}
\DeclareAcronym{fa}{
  short = FA,
  long  = Finite-state Automaton,
  class = abbrev
}
\DeclareAcronym{fas}{
  short = FAs,
  long  = Finite-state Automata,
  class = abbrev
}
\DeclareAcronym{mlp}{
  short = MLP,
  long  = Multi-Layer Perceptron,
  class = abbrev
}
\DeclareAcronym{rnn}{
  short = RNN,
  long  = Recurrent Neural Network,
  class = abbrev
}
\DeclareAcronym{cnn}{
  short = CNN,
  long  = Convolutional Neural Network,
  class = abbrev
}
\DeclareAcronym{gpu}{
  short = GPU,
  long  = Graphics Processing Unit,
  class = abbrev
}
\DeclareAcronym{nlu}{
  short = NLU,
  long  = Natural Language Understanding,
  class = abbrev
}
\DeclareAcronym{nltk}{
  short = NLTK,
  long  = Natural Language Toolkit,
  class = abbrev
}
\DeclareAcronym{lstm}{
  short = LSTM,
  long  = Long Short-Term Memory,
  class = abbrev
}
\DeclareAcronym{wfaw}{
  short = WFA-$\omega$,
  long  = Weighted Finite-state Automaton-$\omega$,
  class = abbrev
}
\DeclareAcronym{wfaws}{
  short = WFA-$\omega$'s,
  long  = Weighted Finite-state Automata-$\omega$,
  class = abbrev
}
\DeclareAcronym{bert}{
  short = BERT,
  long  = Bidirectional Encoder Representations from Transformers,
  class = abbrev
}
\DeclareAcronym{glove}{
  short = GloVe,
  long  = Global Vectors for Word Representation,
  class = abbrev
}
\DeclareAcronym{sopa}{
  short = SoPa,
  long  = Soft Patterns,
  class = abbrev
}
\DeclareAcronym{spp}{
  short = SoPa++,
  long  = Soft Patterns++,
  class = abbrev
}

%----------------------------------------------------------------------------------------
%	MARGIN SETTINGS
%----------------------------------------------------------------------------------------

\geometry{
	paper=a4paper, % Change to letterpaper for US letter
	inner=2.5cm, % Inner margin
	outer=3.8cm, % Outer margin
	bindingoffset=.5cm, % Binding offset
	top=1.5cm, % Top margin
	bottom=1.5cm, % Bottom margin
	%showframe, % Uncomment to show how the type block is set on the page
}

%----------------------------------------------------------------------------------------
%	THESIS INFORMATION
%----------------------------------------------------------------------------------------

\thesistitle{SoPa++: Leveraging explainability from hybridized RNN, CNN and weighted finite-state neural architectures} % Your thesis title, this is used in the title and abstract, print it elsewhere with \ttitle
\supervisor{Dr. Sharid \textsc{Loáiciga} \\ University of Potsdam} % Your supervisor's name, this is used in the title page, print it elsewhere with \supname
\examiner{Mathias \textsc{Müller} \\ University of Zurich} % Your examiner's name, this is not currently used anywhere in the template, print it elsewhere with \examname
\degree{Cognitive Systems: Language, \\ Learning, and Reasoning (M.Sc.)} % Your degree name, this is used in the title page and abstract, print it elsewhere with \degreename
\author{Atreya \textsc{Shankar} (799227)} % Your name, this is used in the title page and abstract, print it elsewhere with \authorname
\addresses{} % Your address, this is not currently used anywhere in the template, print it elsewhere with \addressname

\subject{Computational Linguistics} % Your subject area, this is not currently used anywhere in the template, print it elsewhere with \subjectname
\keywords{} % Keywords for your thesis, this is not currently used anywhere in the template, print it elsewhere with \keywordnames
\university{\href{https://www.uni-potsdam.de/en/university-of-potsdam}{University of Potsdam}} % Your university's name and URL, this is used in the title page and abstract, print it elsewhere with \univname
\department{\href{https://www.uni-potsdam.de/en/ling/index}{Department of Linguistics}} % Your department's name and URL, this is used in the title page and abstract, print it elsewhere with \deptname
\group{\href{http://clp.ling.uni-potsdam.de/}{Foundations of Computational Linguistics Research Group
}} % Your research group's name and URL, this is used in the title page, print it elsewhere with \groupname

\AtBeginDocument{
\hypersetup{pdftitle=\ttitle} % Set the PDF's title to your title
\hypersetup{pdfauthor=\authorname} % Set the PDF's author to your name
\hypersetup{pdfkeywords=\keywordnames} % Set the PDF's keywords to your keywords
}

\begin{document}

\pagestyle{plain} % Default to the plain heading style until the thesis style is called for the body content

%----------------------------------------------------------------------------------------
%	TITLE PAGE
%----------------------------------------------------------------------------------------

\begin{titlepage}
\begin{center}

\vspace*{.06\textheight}
{\scshape\LARGE \univname\par}\vspace{1.5cm} % University name
\textsc{\Large Master's Thesis}\\[0.5cm] % Thesis type

\HRule \\[0.4cm] % Horizontal line
{\huge \bfseries \ttitle\par}\vspace{0.4cm} % Thesis title
\HRule \\[1.5cm] % Horizontal line
 
\begin{minipage}[t]{0.4\textwidth}
\begin{flushleft} \large
\emph{Author:}\\
\href{https://www.linkedin.com/in/atreya-shankar-644352113}{\authorname} % Author name - remove the \href bracket to remove the link
\end{flushleft}
\end{minipage}
\begin{minipage}[t]{0.4\textwidth}
\begin{flushright} \large
\emph{1st Supervisor:} \\
\href{https://sites.google.com/site/loaicigasharid/portada}{\supname} \\[10pt] % Supervisor name - remove the \href bracket to remove the link
\emph{2nd Supervisor:} \\
\href{https://www.uzh.ch/cmsssl/cl/de/people/team/compling/mmueller.html}{\examname} % Supervisor name - remove the \href bracket to remove the link  
\end{flushright}
\end{minipage}\\[3cm]
 
\vfill

\large \textit{A thesis submitted in fulfillment of the requirements\\ for the degree of \degreename}\\[0.3cm] % University requirement text
\textit{in the}\\[0.4cm]
\groupname\\\deptname\\[2cm] % Research group name and department name
 
\vfill

{\large April 12, 2021}\\[4cm] % Date
% \includegraphics[width=5cm]{images/UP.jpg} % University/department logo - uncomment to place it
 
\vfill
\end{center}
\end{titlepage}

%----------------------------------------------------------------------------------------
%	LIST OF CONTENTS/FIGURES/TABLES PAGES
%----------------------------------------------------------------------------------------

{
  \hypersetup{linkcolor=black}
  \pagenumbering{gobble}
  
  \tableofcontents % Prints the main table of contents
  \frontmatter % Use roman page numbering style (i, ii, iii, iv...) for the pre-content pages

  \chapter{Declaration of Originality}

\noindent I, \authorname, declare that this thesis titled \enquote{\ttitle} was
written independently without any unauthorized assistance or sources not given
credit within the work. All words, phrases, passages, data and figures taken
from other sources have been properly cited. No parts of this work in the same
form or a similar form have ever been previously handed in to fulfill an
examination.

\vspace{40pt}

\noindent Signed:\\
\rule[0.5em]{25em}{0.5pt} % This prints a line for the signature
 
\noindent Date: 12.04.2021 \\
\rule[0.5em]{25em}{0.5pt} % This prints a line to write the date

\cleardoublepage

%%% Local Variables: 
%%% mode: latex
%%% TeX-master: "main"
%%% End:
  \chapter{Abstract}

The Thesis Abstract is written here (and usually kept to just this page). The page is kept centered vertically so can expand into the blank space above the title too\ldots

%%% Local Variables: 
%%% mode: latex
%%% TeX-master: "main"
%%% End:
  \chapter{Zusammenfassung}

Im Bereich des maschinellen Lernens (ML) und der Verarbeitung natürlicher
Sprache (NLP) hat es in den letzten Jahren aufgrund der Fortschritte bei tiefen
neuronalen Netzen erhebliche Fortschritte gegeben. Einer der Hauptkritikpunkte
an diesen Black-Box-Modellen war jedoch ihre mangelnde Transparenz und
schwache Erklärbarkeitstechniken. In dieser Arbeit bauen wir auf die
Erklärbarkeit des SoPa-Modells auf, das von Schwartz, Thomson und Smith
(\citeyear{schwartz2018sopa}) vorgeschlagen wurde. Die Stärke des SoPa-Modells
ist, dass es gewichtete endliche Zustandsautomaten in seine neuronale
Architektur einbezieht. Allerdings sind die vorhandenen Erklärungsmethoden
sowohl lokalisiert als auch indirekt. Um diese Einschränkung zu beheben,
entwickelten wir ein modifiziertes SoPa++-Modell, das durch Vereinfachung
globalisierte und direkte Erklärungen bietet. Wir fanden heraus, dass das
SoPa++-Modell einen wettbewerbsfähigen Genauigkeitsbereich von 97,6-98,3$\%$,
effektive Erklärungen durch Vereinfachung und interessante
Klassifizierungserklärungen auf dem Facebook Multilingual Task Oriented Dialog
(FMTOD) Datensatz bietet. Dieses verbesserte SoPa++-Modell zeigt, dass
Black-Box-Erklärbarkeit effektiv erreicht werden kann, ohne einen unpraktischen
Kompromiss bei der Leistung einzugehen.

%%% Local Variables: 
%%% mode: latex
%%% TeX-master: "main"
%%% End:


  \addcontentsline{toc}{chapter}{List of Figures}
  \listoffigures % Prints the list of figures

  \addcontentsline{toc}{chapter}{List of Tables}
  \listoftables % Prints the list of tables

  \addcontentsline{toc}{chapter}{List of Algorithms}
  \listofalgorithms
  
  \printacronyms[name=List of Abbreviations, sort=true] % Print list of abbreviations
  \addcontentsline{toc}{chapter}{List of Abbreviations}
}

%----------------------------------------------------------------------------------------
%	THESIS CONTENT - CHAPTERS
%----------------------------------------------------------------------------------------

\mainmatter % Begin numeric (1,2,3...) page numbering

\pagestyle{thesis} % Return the page headers back to the "thesis" style

% Include the chapters of the thesis as separate files from the Chapters folder
% Uncomment the lines as you write the chapters

\chapter{Introduction}

\label{chapter:introduction}

\section{Motivation}

With the recent trend of increasingly large deep learning models achieving State-Of-The-Art (SOTA) performance on a myriad of Machine Learning (ML) tasks (Figure \ref{fig:nlp_progress}), several studies argue for focused research into Explainable Artificial Intelligence (XAI) to address emerging concerns such as security risks and inductive biases associated with black-box models \citep{doran2017does,townsend2019extracting,danilevsky2020survey,arrieta2020explainable}. Of these studies, \citet[Section 2.2, Page 4]{arrieta2020explainable} provide the following novel definition of XAI based on an extensive literature review of recent XAI research:

\begin{quote}
  \textit{``Given an audience, an \textbf{explainable} Artificial Intelligence is one that produces details or reasons to make its functioning clear or easy to understand.''}
\end{quote}

In addition, \citet{arrieta2020explainable} explore and classify a variety of machine-learning models into transparent and black-box categories depending on their degrees of transparency. Furthermore, they explore taxonomies of post-hoc explainability methods aimed at effectively explaining black-box models. Of high relevance to this study are the local explanations, feature relevance and explanations by simplification post-hoc explainability techniques. 

Through a survey of recent literature on explanations by simplification applied in the Natural Language Processing (NLP) field, we came across several prominent studies employing techniques to simplify black-box neural networks into constituent Finite-State Automata (FSAs) and/or Weighted Finite-State Automata (WFSAs) \citep{schwartz2018sopa,peng2018rational,DBLP:journals/corr/abs-1905-08701,wang2019state,jiang2020cold}.

In this thesis, we build upon the work of \citet{schwartz2018sopa} by further developing their \textbf{So}ft \textbf{Pa}tterns (SoPa) model; which represents a hybridized RNN, CNN and Weighted Finite-State Automaton neural network architecture. We modify the SoPa model by changing key aspects of its architecture which ultimately allows us to conduct effective explanations by simplification; which was not possible with the previous SoPa architecture. We abbreviate this modified model as \textbf{SoPa++}, which signifies an improvement or major modification to the SoPa model. Finally, we evaluate both the performance and explainability of the SoPa++ model on the Facebook Multilingual Task Oriented Dialog data set (FMTOD; \citealt{schuster2018cross}); focusing on the English-language intent classification task.

\begin{figure}[th]
  \centering
  \includegraphics[width=13cm]{pdfs/borrowed/nlp_sota_model_size_progress.pdf}
  \caption{Parameter counts of recently released pre-trained language models which showed competitive or SOTA performance when fine-tuned over a range of NLP tasks; figure taken from \citet{sanh2019distilbert}}
  \label{fig:nlp_progress}
\end{figure}

\section{Research questions}

With the aforementioned modifications to the SoPa architecture and the introduction of the SoPa++ architecture, we aim to answer the following three research questions:

\begin{enumerate}
  \item To what extent does SoPa++ contribute to competitive performance\footnote{We define competitive performance as the scenario where a mean performance metric on a certain data set falls within the range obtained from other recent studies on the same data set} on the FMTOD data set?
  \item To what extent does SoPa++ contribute to effective explanations by simplification on the FMTOD data set?
  \item What interesting and relevant explanations can SoPa++ provide on the FMTOD data set?
\end{enumerate}

\section{Thesis structure}

With the aforementioned research questions, we summarize the structure and contents of this thesis.

\begin{description}[align=left]
  \item [Chapter 1:] Introduce this thesis, its contents and our research questions.
  \item [Chapter 2:] Describe the background concepts utilized in this thesis.
  \item [Chapter 3:] Describe the methodologies pursued in this thesis.
  \item [Chapter 4:] Describe the results obtained from our methodologies.
  \item [Chapter 5:] Discuss the implications of the aforementioned results.
  \item [Chapter 6:] Conclude this thesis by answering the research questions.
  \item [Chapter 7:] Document future work to expand on our research questions.
\end{description}

%%% Local Variables: 
%%% mode: latex
%%% TeX-master: "main"
%%% End: 

% LocalWords:  SOTA XAI explainability NLP Automata FSAs WFSAs FMTOD th nlp pre
% LocalWords:  sota tterns

\chapter{Background concepts}

\label{background}

\section{Explainable artificial intelligence}

In this section, we lay out background concepts for Explainable Artificial Intelligence (XAI) which have been largely adopted from \citet{arrieta2020explainable}. The study is particularly helpful for us since it summarizes the findings of approximately 400 XAI contributions and presents these findings in the form of well-defined concepts and taxonomies. In addition, the study discusses future directions of XAI research. We start off by providing definitions from the study, along with accompanying remarks taken either directly from the study or paraphrased for brevity.

\subsection{Transparency}

\begin{definition}[Transparency]
  A model is considered to be transparent if by itself it is understandable. Since a model can feature different degrees of understandability, transparent models are divided into three categories: simulatable models, decomposable models and algorithmically transparent models. 
\end{definition}

\begin{remark}
  \textit{Simulatability} denotes the ability of a model of being simulated or thought about strictly by a human, hence complexity takes a dominant place in this class.
\end{remark}

\begin{remark}
  \textit{Decomposability} stands for the ability to explain each of the parts of a model (input, parameter and calculation).
\end{remark}

\begin{remark}
  \textit{Algorithmic transparency} deals with the ability of the user to understand the process followed by the model to produce any given output from its input data.
\end{remark}

\begin{remark}
  A model is considered transparent if it falls into one or more of the aforementioned transparency categories.
\end{remark}

\begin{remark}
  If a model cannot satisfy the requirements of being transparent, then it is classified as a \textit{black-box} model. 
\end{remark}

Examples of well-known transparent Machine Learning (ML) models are linear/logistic regressors, decision trees and rules-based learners. Similarly, common examples of non-transparent or black-box ML models are tree ensembles and deep neural networks. \citet{arrieta2020explainable} provide extensive justifications using the aforementioned three criteria in conducting model classifications into the transparent and black-box categories. We would direct the reader to their study for a full analysis and justification of these classifications.

\subsection{Explainability and XAI}

\begin{definition}[Explainability]
  Explainability is associated with the notion of explanation as an interface between humans and a decision maker that is, at the same time, both an accurate proxy of the decision maker and comprehensible to humans. 
\end{definition}

\begin{definition}[Explainable Artificial Intelligence]
  Given an audience, an \textbf{explainable} Artificial Intelligence is one that produces details or reasons to make its functioning clear or easy to understand.
\end{definition}

\citet{arrieta2020explainable} observe that black-box ML models are increasingly being employed to make important predictions in critical contexts, citing high-risk areas such as precision medicine and autonomous vehicles. Of particular relevance to the field of Natural Language Processing (NLP), the study notes a myriad of issues related to inductive biases within training data sets and the ethical issues involved with using black-box models trained on such data sets. As a result, they describe the increased demand for transparency in black-box ML models from the various stakeholders in Artificial Intelligence (AI). In addition, \citet{arrieta2020explainable} concretize the presence of a target audience for XAI; implying that different XAI techniques should be employed for different target audiences. In their study, they provide examples of target audiences such as domain experts, end-users and managers (Figure \ref{fig:xai_target_audience}).

\begin{figure}[t]
  \centering
  \includegraphics[trim={0.1cm 0.1cm 0.1cm 0.1cm},clip,width=14cm]{pdfs/xai_target_audience}
  \caption{Examples of various target audiences in XAI \citep{arrieta2020explainable}}
  \label{fig:xai_target_audience}
\end{figure}

\subsection{Explainability techniques}

Based on the aforementioned classification of ML models into transparent and black-box models, \citet{arrieta2020explainable} expound on explainability techniques for each of these model types. Due to their transparent nature, the study states that transparent ML models are usually explainable in themselves to most target audiences and therefore usually do not require any external technique to extract explanations. The study does however highlight some target audiences, such as non-expert users, who may require external explainability techniques such as model output visualizations in order to explain the inner workings of transparent ML models.

For the case of non-transparent or black-box models, \citet{arrieta2020explainable} argue that separate or external techniques must be utilized in order to reasonably explain these models. Such explainability techniques are referred to in the study as post-hoc explainability techniques; which is derived from the idea that explanations for such models are usually extracted after the modelling procedure. Notable examples of post-hoc explainability techniques include local explanations, feature relevance and explanations by simplification. Below we provide definitions for these methods, which have been adapted from \citet{arrieta2020explainable}:

\begin{definition}[Local explanations]
  Local explanations tackle explainability by segmenting the solution space and giving explanations to less complex solution subspaces that are relevant for the whole model.
\end{definition}

\begin{remark}
  Two well-known examples of local explainability techniques are Local Interpretable Model-Agnostic Explanations (LIME; \citealt{lime}) and G-REX \citep{konig2008g}.
\end{remark}

\begin{definition}[Feature relevance]
  Feature relevance explanation methods clarify the inner functioning of a model by computing a relevance score for its managed variables. These scores quantify the affection (sensitivity) a feature has upon the output of the model.
\end{definition}

\begin{remark}
  A well-known feature relevance explainability technique is known as the Shapley Additive Explanations (SHAP; \citealt{lundberg2017unified}). Another similar feature relevance explainability technique is known as the occlusion sensitivity method \citep{zeiler2014visualizing}.
\end{remark}

\begin{definition}[Explanations by simplification]
  Explanations by simplification collectively denote those techniques in which a whole new system is rebuilt based on the trained model to be explained. This new, simplified model usually attempts at optimizing its resemblance to its antecedent functioning, while reducing its complexity, and keeping a similar performance score.
\end{definition}

\begin{remark}
  We hereby refer to the original black-box model as an \textit{oracle} model and the simplified version of the model as the \textit{proxy} model. Furthermore, we qualify that all proxy models must be able to globally approximate their respective oracle models. This is in constrast to local explanations which only approximate subsets of oracle models.
\end{remark}

\begin{remark}
  \citet{bastani2017interpretability} and \citet{tan2018distill} are examples of studies that extract and distill simpler proxy models from complex oracle models.
\end{remark}

Through a survey of recent literature on explanations by simplification applied in the Natural Language Processing (NLP) field, we came across several prominent studies employing explanations by simplification to simplify black-box neural networks into constituent Finite-State Automata (FSA) and/or Weighted Finite-State Automata (WFSA) \citep{schwartz2018sopa,peng2018rational,DBLP:journals/corr/abs-1905-08701,wang2019state,jiang2020cold}. We expound more on WFSA and \citet{schwartz2018sopa} in Sections \ref{wfsa} and \ref{sopa} respectively.

\subsection{Key insights}

\section{Straight-through estimator}

\section{Weighted finite-state automata}

\label{wfsa}

\begin{definition}[Semiring; \citealt{kuich1986linear}]
  A semiring is a set $\mathbb{K}$ along with two binary associative operations $\oplus$ (addition) and $\otimes$ (multiplication) and two identity elements: $\bar{0}$ for addition and $\bar{1}$ for multiplication. Semirings require that addition is commutative, multiplication distributes over addition, and that multiplication by $\bar{0}$ annihilates, i.e., $\bar{0} \otimes a = a \otimes \bar{0} = \bar{0}$.

\begin{remark}
  Semirings follow the following generic notation: $\langle \mathbb{K}, \oplus, \otimes, \bar{0}, \bar{1} \rangle$.
\end{remark}

\begin{remark}
  A simple and common semiring is the real or sum-product semiring: $\langle \mathbb{R}, +, \times, 0, 1 \rangle$. Two important semirings for this thesis are shown below.
\end{remark}

\begin{remark}
  \textbf{Max-sum} semiring: $\langle \mathbb{R} \cup \{-\infty\}, \text{max}, +, -\infty, 0 \rangle$
\end{remark}

\begin{remark}
  \textbf{Max-product} semiring: $\langle \mathbb{R}_{>0} \cup \{-\infty\}, \text{max}, \times, -\infty, 1 \rangle$
\end{remark}

\end{definition}

\begin{definition}[Weighted finite-state automaton; \citealt{peng2018rational}]
  A weighted finite-state automaton over a semiring $\mathbb{K}$ is a 5-tuple $\mathcal{A} = \langle \Sigma, \mathcal{Q}, \mathcal{T}, \lambda, \rho \rangle$, with:

  \begin{itemize}
    \itemsep0em 
    \item[--] a finite input alphabet $\Sigma$;
    \item[--] a finite state set $\mathcal{Q}$;
    \item[--] transition weights $\mathcal{T}: \mathcal{Q} \times \mathcal{Q} \times (\Sigma \cup \{\epsilon\}) \rightarrow \mathbb{K}$;
    \item[--] initial weights $\lambda: \mathcal{Q} \rightarrow \mathbb{K}$; 
    \item[--] and final weights $\rho: \mathcal{Q} \rightarrow \mathbb{K}$.
  \end{itemize}

  \begin{remark}
    $\epsilon \notin \Sigma$ refers to special $\epsilon$-transitions that may be taken without consuming any input.
  \end{remark}

  \begin{remark}
    Self-loop transitions in $\mathcal{A}$ refer to special transitions which consume an input while staying at the same state.
  \end{remark}
  
  \begin{remark}
    $\Sigma^{*}$ refers to the (possibly infinite) set of all strings over the alphabet $\Sigma$.
  \end{remark}
   
\end{definition}

\begin{definition}[Path score; \citealt{peng2018rational}]

  Let $\pmb{\pi} = \langle \pi_1, \pi_2, \dots, \pi_n \rangle$ be a sequence of adjacent transitions in $\mathcal{A}$, with each $\pi_i = \langle q_i, q_{i+1}, z_i \rangle \in \mathcal{Q} \times \mathcal{Q} \times (\Sigma \cup \{\epsilon\})$. The path $\pmb{\pi}$ derives the $\epsilon$-free string $\pmb{x} = \langle x_1, x_2, \dots, x_m \rangle \in \Sigma^{*}$; which is a substring of the $\epsilon$-containing string $\pmb{z} = \langle z_1, z_2, \dots, z_n \rangle \in (\Sigma \cup \{\epsilon\})^{*}$. $\pmb{\pi}$'s score in $\mathcal{A}$ is given by:
  
\end{definition}

\begin{equation}
  \mathcal{A}[\pmb{\pi}] = \lambda(q_1) \otimes \Bigg( \bigotimes_{i=1}^n \mathcal{T}(\pi_i) \Bigg) \otimes \rho(q_{n+1})
\end{equation}

\begin{definition}[String score; \citealt{peng2018rational}]

Let $\Pi(\pmb{x})$ denote the set of all paths in $\mathcal{A}$ that derive $\pmb{x}$. Then the string score assigned by $\mathcal{A}$ to string $\pmb{x}$ is given by:
  
\end{definition}

\begin{equation}
  \mathcal{A}[\![\pmb{x}]\!] = \bigoplus_{\pmb{\pi} \in \Pi(\pmb{x})} \mathcal{A}[\pmb{\pi}]
\end{equation}

\begin{remark}
  Since $\mathbb{K}$ is a semiring, $\mathcal{A}[\![\pmb{x}]\!]$ can be efficiently computed using the Forward algorithm \citep{baum1966statistical}. Its dynamic program is summarized below without $\epsilon$-transitions for simplicity. $\Omega_i(q)$ gives the aggregate score of all paths that derive the substring $\langle x_1, x_2, \dots, x_i \rangle$ and end in state $q$:
 
\begin{subequations}
  \begin{align}
    \Omega_0(q) &= \lambda(q) \\
    \Omega_{i+1}(q) &= \bigoplus_{q' \in \mathcal{Q}} \Omega_i(q') \otimes \mathcal{T}(q',q,x_i)  \\
    \mathcal{A}[\![\pmb{x}]\!] &= \bigoplus_{q \in \mathcal{Q}} \Omega_n(q) \otimes \rho(q)
  \end{align}
\end{subequations}

\end{remark}

\begin{remark}
  The Forward algorithm can be generalized to any semiring \citep{eisner2002parameter} and has a runtime of $O(|Q|^3 + |Q|^2|\pmb{x}|)$ \citep{schwartz2018sopa}; notably with a linear runtime with respect to the length of the input string $\pmb{x}$.
\end{remark}

\begin{remark}
  A special case of Forward is the Viterbi algorithm, where the addition $\oplus$ operator is contrained to the maximum function \citep{viterbi1967error}. Viterbi therefore returns the highest scoring path $\pmb{\pi}$ that derives the input string $\pmb{x}$.
\end{remark}

\section{Soft patterns}
\label{sopa}

%%% Local Variables: 
%%% mode: latex
%%% TeX-master: "main"
%%% End: 

\chapter{Data and methodologies}

\label{chapter:methodologies}

In this chapter, we describe the FMTOD data set in greater detail and
survey the performance of other studies that addressed this
data set and its tasks. Following this, we introduce the core content of this
thesis by describing the methodologies used for answering our three research
questions. Comprehensive source code reflecting our methodologies can be found
in our public GitHub
repository\footnote{https://github.com/atreyasha/spp-explainability}.

\section{Facebook Multilingual Task Oriented Dialog}

\citet{schuster-etal-2019-cross-lingual} originally released the Facebook
Multilingual Task Oriented Dialog (FMTOD) data set to encourage research in
cross-lingual transfer learning for dialogue-oriented Natural Language
Understanding (NLU) tasks; specifically from from high-resource to low-resource
languages. The authors released the FMTOD data set with English as the
high-resource language providing $\sim$43k utterances, and Spanish and Thai as
low-resource languages providing a total of $\sim$14k utterances. Furthermore,
they streamlined the data set towards two key tasks; namely intent
classification and textual slot filling. In this thesis, we focus solely on the
English language intent classification task in the FMTOD data set; which entails
a multi-label sequence classification task with a total of 12 classes from
alarm, reminder and weather-related domains. For brevity, we refer to the FMTOD
English language intent classification data set as the FMTOD data set.

We chose to work with the FMTOD data set since it is both a recently released
and well-studied data set
\citep{schuster-etal-2019-cross-lingual,zhang2019joint,zhang-etal-2020-intent}.
We focus on the English language intent classification task since it is a
relatively straightforward task which allows us to place a greater focus on
performance and explainability. Furthermore, the English language subset entails
the highest resources in the FMTOD data set. Finally, we find the FMTOD data
set's intent classification task especially attractive because it allows us
to test the SoPa++ model on a multi-class NLU problem; which is significantly
different from the focus on binary classification sentiment detection tasks in
SoPa. 

\subsection{Preprocessing}

Given that we are handling text-based data in the FMTOD data set, it is
necessary to preprocess this data first before proceeding with any modeling
steps. We enumerate our preprocessing steps below:

\begin{enumerate}{}
  \item We convert all FMTOD text samples into a lowercased format. This assists
  in simplifying and normalizing the textual data.
  \item Next, we remove data duplicates within the provided training, validation and test
  data partitions.
  \item Finally, we remove data duplicates which overlap between partitions.
  During this step, we do not remove any cross-partition duplicates from the
  test partition in order to keep it as similar as possible to the original test
  partition. This comes into importance later when we compare performance metrics
  on the test set with other studies.
\end{enumerate}

\begin{figure}[t!]
  \centering
  \includegraphics[width=14cm]{pdfs/generated/fmtod_summary_statistics.pdf}
  \caption{Frequency distribution of the preprocessed FMTOD data set by
    classes and partitions}
  \label{fig:fmtod}
\end{figure}

\begin{table}[t!]
  \centering
  \begin{tabular}{lllll}
    \toprule
    Class and description & Train & Validation & Test & $\Sigma$ \\
    \midrule
    0: \texttt{alarm/cancel\_alarm} & 1157 & 190 & 444 & 1791 \\
    1: \texttt{alarm/modify\_alarm} & 393 & 51 & 122 & 566 \\
    2: \texttt{alarm/set\_alarm} & 3584 & 596 & 1236 & 5416 \\
    3: \texttt{alarm/show\_alarms} & 619 & 83 & 212 & 914 \\
    4: \texttt{alarm/snooze\_alarm} & 228 & 49 & 89 & 366 \\
    5: \texttt{alarm/time\_left\_on\_alarm} & 233 & 30 & 81 & 344 \\
    6: \texttt{reminder/cancel\_reminder} & 662 & 114 & 284 & 1060 \\
    7: \texttt{reminder/set\_reminder} & 3681 & 581 & 1287 & 5549 \\
    8: \texttt{reminder/show\_reminders} & 474 & 82 & 217 & 773 \\
    9: \texttt{weather/check\_sunrise} & 63 & 13 & 25 & 101 \\
    10: \texttt{weather/check\_sunset} & 88 & 11 & 37 & 136 \\
    11: \texttt{weather/find} & 9490 & 1462 & 3386 & 14338 \\[5pt]
    \hline \hline \\[-10pt]
    $\Sigma$ & 20672 & 3262 & 7420 & 31354 \\
    \bottomrule
  \end{tabular}
  \caption{Frequency of the preprocessed FMTOD data set classes grouped by
    partitions; $\Sigma$ signifies the cumulative frequency statistic}
  \label{tab:fmtod}
\end{table}

During the preprocessing phase, many data duplicates were encountered and
correspondingly removed. Some of these duplicates observed were already present
in the original FMTOD data set, with additional duplicates being created from
the initial lowercasing step. After preprocessing, we obtain a lowercased
variant of the FMTOD data set with strictly unique data partitions. In the next
section, we describe the summary statistics of the preprocessed FMTOD data set.

\begin{table}[t!]
  \centering
  \begin{threeparttable}
    \begin{tabular}{lll}
      \toprule
      Class and description & Utterance length$^{\dagger}$ & Example$^{\ddagger}$ \\
      \midrule
      0: \texttt{alarm/cancel\_alarm} & 5.6 $\pm$ 1.9 & cancel weekly alarm \\
      1: \texttt{alarm/modify\_alarm} & 7.1 $\pm$ 2.5 & change alarm time \\
      2: \texttt{alarm/set\_alarm} & 7.5 $\pm$ 2.5 & please set the new alarm \\
      3: \texttt{alarm/show\_alarms} & 6.9 $\pm$ 2.2 & check my alarms. \\
      4: \texttt{alarm/snooze\_alarm} & 6.1 $\pm$ 2.1 & pause alarm please \\
      5: \texttt{alarm/time\_left\_on\_alarm} & 8.6 $\pm$ 2.1  & minutes left on my alarm \\
      6: \texttt{reminder/cancel\_reminder} & 6.6 $\pm$ 2.2 & clear all reminders. \\
      7: \texttt{reminder/set\_reminder} & 8.9 $\pm$ 2.5 & birthday reminders \\
      8: \texttt{reminder/show\_reminders} & 6.8 $\pm$ 2.2 & list all reminders \\
      9: \texttt{weather/check\_sunrise} & 6.7 $\pm$ 1.7 & when is sunrise \\
      10: \texttt{weather/check\_sunset} & 6.7 $\pm$ 1.7 & when is dusk \\
      11: \texttt{weather/find} & 7.8 $\pm$ 2.3 & jacket needed? \\[5pt]
      \hline \hline \\[-10pt]
      $\mu$ & 7.7 $\pm$ 2.5 & \textemdash \\
      \bottomrule
    \end{tabular}
    \begin{tablenotes}[flushleft]
      \footnotesize
      \item $^{\dagger}$Summary statistics follow the mean $\pm$
      standard-deviation format
      \item $^{\ddagger}$Short and simple examples were chosen for brevity and
      formatting purposes
    \end{tablenotes}
  \end{threeparttable}
  \caption{Utterance length summary statistics and examples for the preprocessed
    FMTOD data set; $\mu$ signifies the cumulative summary statistic}
  \label{tab:fmtod_examples}
\end{table}

\begin{table}[t!]
  \centering \def\arraystretch{1.3}
  \begin{tabular}{L{0.27\linewidth} L{0.45\linewidth} l}
    \toprule
    Study & Summary & Accuracy \\
    \midrule
    \citet{schuster-etal-2019-cross-lingual} & BiLSTM jointly trained on both the slot filling and intent classification tasks & 99.1$\%$ \\
    \citet{zhang2019joint} & BERT along with various decoders jointly fine-tuned on both the slot filling and intent classification tasks & 96.6--98.9$\%$ \\
    \citet{zhang-etal-2020-intent} & RoBERTa and XLM-RoBERTa fine-tuned on the English language and multilingual intent classification tasks along with WikiHow pre-training & 99.3--99.5$\%$ \\
    \bottomrule
  \end{tabular}
  \caption{Studies that addressed the FMTOD English language intent
    classification task along with their relevant summaries and accuracy
    ranges}
  \label{tab:fmtod_results}
\end{table}

\subsection{Summary statistics}

\label{section:fmtod_summary}

In regards to summary statistics of the preprocessed FMTOD data set, Figure
\ref{fig:fmtod} provides a visualization of the frequency distribution in the
data set grouped by classes and partitions; while Table \ref{tab:fmtod} shows
the same summary statistics in a tabular form with explicit frequencies. Based
on the summary statistics, we can observe that the preprocessed FMTOD data set
is significantly imbalanced with $\sim$45$\%$ of samples falling into Class 11
alone. We take this observation into consideration in later sections and apply
fixes to mitigate this data imbalance. In addition, we observe from Table
\ref{tab:fmtod_examples} that input utterances in the preprocessed FMTOD data
set are generally short; with a mean input utterance length of 7.7 and a
standard deviation of 2.5 tokens. Utterance length summary statistics were
computed with the assistance of NLTK's default \texttt{Treebank} word tokenizer
\citep{bird-loper-2004-nltk}.

\subsection{Performance range}

\label{section:fmtod_performance}

Several recent studies have addressed the FMTOD English language intent
classification task using a variety of deep neural networks such as BiLSTMs and
large language models such as XLM-RoBERTa
\citep{schuster-etal-2019-cross-lingual,zhang2019joint,zhang-etal-2020-intent}.
Table \ref{tab:fmtod_results} summarizes these studies along with their reported
accuracy scores on the FMTOD English language intent classification task. Based
on the presented results, we can observe that the competitive accuracy range for
the FMTOD English language intent classification task is 96.6-99.5$\%$.

\section{SoPa++}

We now introduce the main contribution of this thesis: the SoPa++ model.
Etymologically, SoPa++ derives its name from the variable increment operator
\texttt{"++"} used in programming languages such as C and Java. In essence, the
name SoPa++ signifies an improvement or major modification to the SoPa model.
Some of the major modifications from SoPa to SoPa++ include the utility of
strict linear-chain WFA-$\omega$'s over linear-chain WFAs, replacement of the
MLP in SoPa with quantized and transparent hidden layers and the introduction of
a new explanations by simplification post-hoc explainability technique which
leverages on the aforementioned modifications in SoPa++'s neural architecture.

\subsection{Strict linear-chain WFA-$\omega$'s}

As mentioned in Section \ref{section:sopa}, \citet{schwartz2018sopa} constructed
the SoPa model with an ensemble of linear-chain WFAs which permitted both
$\epsilon$ and self-loop transitions. As noted in Section \ref{section:sopa_cg},
$\epsilon$ and self-loop transitions are useful constructs in abstracting WFAs
and allowing them to consume variable length strings. However, based on
experimentation during our development phase; we observed a key concern that the
highest scoring substrings in the linear-chain WFAs in SoPa tended to have a
large variation of string lengths due to the effect of both
$\epsilon$-transitions and self-loops. We believe that this reduces the impact
of SoPa's explainability methods due to a lack of consistency in the lengths
of highest scoring paths and substrings. As a result, the first change we decided
for was to remove both $\epsilon$ and self-loop transitions in constituent WFAs
or patterns. With this change, we could at least ensure that each WFA would
always consume strings of fixed lengths.

However, consuming strings of fixed lengths could also be seen as a form of
overfitting in the model; since a model could simply memorize short strings or
phrases and would not necessarily incorporate any form of generalization. To
address this concern, we include a wildcard transition which we define here as a
$\omega$-transition. Allowing for such a transition was only natural since
wildcards are already crucial parts of regular expressions; which as we
mentioned are equivalent to FAs. To provide formalisms for this modification, we
provide the following definition for a WFA-$\omega$.

\begin{definition}[Weighted finite-state automaton-$\omega$]
  \label{def:wfa_w}
  A weighted finite-state automaton-$\omega$ over a semiring $\mathbb{K}$ is a
  5-tuple $\mathcal{A} = \langle \Sigma, \mathcal{Q}, \bm{\Gamma}, \bm{\lambda}, \bm{\rho}
  \rangle$, with:

  \begin{itemize}
  \itemsep0em
    \item[--] a finite input alphabet $\Sigma$;
    \item[--] a finite state set $\mathcal{Q}$;
    \item[--] transition matrix $\bm{\Gamma}: \mathcal{Q} \times \mathcal{Q} \times (\Sigma \cup \{\omega\}) \rightarrow \mathbb{K}$;
    \item[--] initial vector $\bm{\lambda}: \mathcal{Q} \rightarrow \mathbb{K}$;
    \item[--] and final vector $\bm{\rho}: \mathcal{Q} \rightarrow \mathbb{K}$.
  \end{itemize}

  \begin{remark}
    An $\omega$ transition is equivalent to a wildcard transition, which
    consumes an arbitrary token input and moves to the next state
  \end{remark}

  \begin{remark}
    Besides the inclusion of the $\omega$-transition and removal of the
    $\epsilon$-transition, a WFA-$\omega$ has all of the same characteristics
    as the WFA defined in Definition \ref{def:wfa}.
  \end{remark}
\end{definition}

Comparing with the linear-chain WFAs used in \citet{schwartz2018sopa} as per
Section \ref{section:sopa_lc_wfa}, our \textit{strict} linear-chain WFA-$\omega$
is similarly alloted a sequence of $|\mathcal{Q}|$ states. However, each state
$i$ in the linear-chain WFA-$\omega$ only has two possible outgoing transitions;
namely a \textbf{$\bm{\omega}$-transition} which consumes an arbitrary input
token and transitions to state $i+1$ and a \textbf{main-path transition} which
consumes a specific token and transitions to state $i+1$. Because of the
elimination of self-loop transitions, we refer to our linear-chain WFA-$\omega$
as strict linear-chain WFA-$\omega$ as per Remark
\ref{rmk:strict_linear_chain}. Similar to \citet{schwartz2018sopa}, we utilize
only the max-sum and max-product semirings in our strict linear-chain
WFA-$\omega$'s.

\begin{figure}[t!]
  \centering
  \includegraphics[width=14cm]{pdfs/generated/w_nfa_linear_chain/main.pdf}
  \caption{Visualization of a strict linear-chain NFA with
    $\omega$ (blue) and main-path (black) transitions}
  \label{fig:omega_fa}
\end{figure}

Next, we provide a mathematical formulation of the modified transition matrix
$\bm{\Gamma}$ in our strict linear-chain WFA-$\omega$. Here, $\bm{\Gamma}(x)$ represents a
$|Q|\times|Q|$ matrix containing transition scores when consuming an input token
$x$. $[\bm{\Gamma}(x)]_{i,j}$ corresponds to the cell value in $\bm{\Gamma}(x)$ for row
$i$ and column $j$ and represents the transition score when consuming token $x$
and transitioning from state $i$ to $j$.

\begin{equation}
  \label{eq:spp_transition_matrix_main}
  [\bm{\Gamma}(x)]_{i,j} =
  \begin{cases}
    \bm{w}_i \cdot \bm{v}_x + b_i  & \text{if } j = i + 1 \text{ (main-path transition),} \\
    \bar{0} & \text{otherwise.}
  \end{cases}
\end{equation}

Here, $\bm{w}_i$ and $b_i$ are learnable vectors and scalar biases
parameterizing transitions out of state $i$ to state $i+1$. $\bm{v}_x$
represents the word embedding for token $x$ and $\bar{0}$ represents the zero
value in the semiring used as per Definition \ref{def:semiring}. Similarly,
$\omega$-transitions are parameterized with the following representation in
$\bm{\Gamma}$:

\begin{equation}
  \label{eq:spp_transition_matrix_omega}
  [\bm{\Gamma}(\omega)]_{i,j} =
  \begin{cases}
    c_i  & \text{if } j = i + 1 \text{ ($\omega$-transition)} \\
    \bar{0} & \text{otherwise.}
  \end{cases}
\end{equation}

Here, $c_i$ represents a learnable scalar bias for $\omega$-transitions out of
state $i$ to state $i+1$. Finally as per \citet{schwartz2018sopa}, we fix the
initial vector $\bm{\lambda} = [\bar{1}, \bar{0}, \ldots, \bar{0}]$ and the final
vector $\bm{\rho} = [\bar{0}, \bar{0}, \ldots, \bar{1}]$, where $\bar{1}$ and
$\bar{0}$ represent the one and zero values specified in the semiring as per
Definition \ref{def:semiring}. The time-complexity of the Viterbi algorithm to
compute the string score for our linear-chain WFA-$\omega$'s is
$O(|Q||\bm{x}|)$, where $|\mathcal{Q}|$ refers to the number of states and
$|\bm{x}|$ refers to the length of the input string.

Ultimately, the introduction of a strict linear-chain WFA-$\omega$ allows us to
attain fixed string length consumption with an added layer of generalization
because of the introduction of wildcards. An example of a strict linear-chain
NFA extracted from a strict linear-chain WFA-$\omega$ is shown in Figure
\ref{fig:omega_fa}. Interestingly, we can observe that this NFA corresponds to
the Perl-compatible regular expression \texttt{``what a (great|entertaining)
  [\^{}\textbackslash s]+ !''}, where
\texttt{[\^{}\textbackslash s]+} refers to any set of consecutive
characters which are not separated by a space character. We can therefore infer
that a $\omega$-transition in a FA corresponds to the
\texttt{[\^{}\textbackslash s]+} regular expression term.

\begin{algorithm}[t!]
  \small
  \caption{Strict linear-chain WFA-$\omega$ document score$^*$}
  \label{algo:lc_wfa_w_document_score}
  \begin{algorithmic}[1]
    \Require{Strict linear-chain WFA-$\omega$ (denoted as $\mathcal{A}$) and document $\bm{y}$}
    \Ensure{Document score $s_{\text{doc}}(\bm{y})$}
    \Statex
    \Function{docscore}{$\mathcal{A}, \bm{y}$}
    \State $\bm{h}_0 \gets \big[\bar{1}, -\infty, \ldots, -\infty\big]$ \Comment{Create
      initial hidden state vector $\bm{h}_0: \mathcal{Q} \rightarrow \mathbb{K}$}
    \For{$i \gets 1,2,\ldots,\bm{|y|}$} \Comment{Sequentially iterate over each
      token $y_i \in \bm{y}$}
    \State $\bm{m} \gets \big[[\bm{\Gamma}(y_i)]_{1,2}, [\bm{\Gamma}(y_i)]_{2,3}, \ldots,
    [\bm{\Gamma}(y_i)]_{|\mathcal{Q}|-1,|\mathcal{Q}|}\big]$ \Comment{Main-path scores for token $y_i$}
    \State $\bm{\omega} \gets \big[[\bm{\Gamma}(\omega)]_{1,2}, [\bm{\Gamma}(\omega)]_{2,3}, \ldots,
    [\bm{\Gamma}(\omega)]_{|\mathcal{Q}|-1,|\mathcal{Q}|}\big]$
    \Comment{State-wise wildcard scores}
    \State $\bm{m'} \gets \bm{m} \otimes \bm{h}_{i-1}[:-1]$ \Comment{Path
      score with main-path transitions$^{\dagger}$}
    \State $\bm{\omega'} \gets \bm{\omega} \otimes \bm{h}_{i-1}[:-1]$ \Comment{Path
    score with $\omega$ transitions$^{\dagger}$}
    \State $\bm{h}_{i} \gets [\bar{1}] \mathbin\Vert \max(\bm{m'}, \bm{w'})$
    \Comment{Concatenate $\bar{1}$ to maximum of $\bm{m'}$ and $\bm{\omega'}$}    
    \EndFor
    \State $s_{\text{doc}}(\bm{y}) \gets  \max_{i \in 1,2,...,|\bm{y}|}
    \bm{h}_{i}[-1]$
    \Comment{Get maximum of hidden vector final states$^{\ddagger}$}
    \State \Return $s_{\text{doc}}(\bm{y})$
    \EndFunction
  \end{algorithmic}
  \algcomment{
    $^*$$\otimes$ and $\bar{1}$ are
    derived from max-based semirings where all semiring operations are element-wise \\
    $^{\dagger}\bm{h}_{i}[:-1]$ follows the Python indexing syntax
    and implies keeping all elements of $\bm{h}_{i}$ except the last \\
    $^{\ddagger}\bm{h}_{i}[-1]$ follows the Python indexing syntax and implies
    retrieving the last element of the vector $\bm{h}_{i}$
  }
\end{algorithm}

\subsection{Document score}

Similar to \citet{schwartz2018sopa}, SoPa++ was intended to compute scores for
entire documents and not just fixed-length strings using the strict linear-chain
WFA-$\omega$. To achieve this, we propose Algorithm
\ref{algo:lc_wfa_w_document_score} to compute the document score
$s_{\text{doc}}(\bm{y})$ for an arbitrary document $\bm{y}$. Here, we score all
consecutive substrings in a document $\bm{y}$ using either the max-sum or
max-product semirings assisted with the Viterbi algorithm. Following from this,
the document score $s_{\text{doc}}(\bm{y})$ for an arbitrary document $\bm{y}$
would reflect the highest path score which corresponds to a substring in
document $\bm{y}$.

\subsection{TauSTE}

In Section \ref{section:ste}, we described the concept of a STE in quantized
neural networks and explained how STEs function in both their forward and
backward passes. Furthermore, we provided a motivation as to why STEs and other
quantized activation functions are of interest; for example in relation to
computational savings linked to low-precision computing. In SoPa++, we make use
of a variant of the STE activation function which we define here as the Tau
Straight-through Estimator (TauSTE).

\begin{equation}
  \label{eq:tau_ste_forward}
  \text{TauSTE}(x)=
  \begin{cases}
    1 & x \in (\tau, +\infty) \\
    0 & x \in (-\infty, \tau]
  \end{cases}
\end{equation}

\begin{equation}
  \label{eq:tau_ste_backward}
  \text{TauSTE}'(x)=
  \begin{cases}
    1 & x \in  (1, +\infty) \\
    x & x \in [-1, 1] \\
    -1 & x \in (-\infty, -1) \\
  \end{cases}
\end{equation}

\begin{figure}[t!]
  \centering
  \includegraphics[width=14cm]{pdfs/generated/tau_ste_applied/main.pdf}
  \caption{Visualization of the TauSTE's forward and backward passes}
  \label{fig:tau_ste}
\end{figure}

A visualization of the TauSTE's forward and backward passes is shown in Figure
\ref{fig:tau_ste}. As we can see, there are two key changes from the vanilla STE
to the TauSTE. Firstly, the threshold for activation in the forward function is
now governed by a $\tau$-threshold such that $\tau \in \mathbb{R}$. This was
done to allow for some degree of freedom in deciding the activation threshold.
Secondly, the backward pass returns the identity function on $x \in [-1,1]$ and
remains fixed at -1 or +1 beyond these limits. This restriction was placed to
ensure that gradients do not blow up in size.

We chose to use the TauSTE because we believed it could prove useful for the
explainability purposes of our SoPa++ model. This is mainly because the TauSTE
(or even the STE) activation function simplifies continuous inputs into discrete
outputs; thereby reducing the information content of the signal it receives. In
later sections, we show how we capitalize on this reduction in signal
information in our SoPa++ model for our explanations by simplification post-hoc
explainability technique.

\subsection{Computational graph}

With the core modifications in the SoPa++ model explained, we now shift to
describing the computational graph or forward pass of the SoPa++ model by
referring to its various neural components. This description is linked to the
visualization of the computational graph of SoPa++ in Figure \ref{fig:spp_cg}.
Firstly, we utilize NLTK's default \texttt{Treebank} word tokenizer
\citep{bird-loper-2004-nltk} to conduct tokenization of input utterances into
word-level tokens. Next, we pad input utterances with special \texttt{[START]}
and \texttt{[END]} tokens at the start and end indices of the utterances to
signify the location where the utterance begins and ends. Finally, we utilize
GloVe 6B 300-dimensional uncased word-level embeddings
\citep{pennington2014glove} to project the input tokens to continuous numerical
spaces. 

\begin{figure}[t!]
  \centering
  \includegraphics[width=15cm]{pdfs/generated/spp_computational_graph/main.pdf}
  \caption{Visualization of the SoPa++ computational graph}
  \label{fig:spp_cg}
\end{figure}

Following this, we use our ensemble of $m \in \mathbb{N}$ strict linear-chain
WFA-$\omega$'s to traverse the input utterance and provide individual document
scores for this utterance; as prescribed by Algorithm
\ref{algo:lc_wfa_w_document_score}. When processing each input token, we monitor
the end states of each of the $m$ strict linear-chain WFA-$\omega$'s and
max-pool the score present in this state. In the edge case that an input string
was too short for the end state of a WFA-$\omega$ to register a non-$\bar{0}$
score, we simply discard this score in further analysis. This description so far
corresponds to the lower half of Figure \ref{fig:spp_cg}. After max-pooling
scores from all the $m$ strict linear-chain WFA-$\omega$'s, we then pass this
collection of pattern scores for further processing using SoPa++'s hidden neural
components; which can be seen in the upper portion of Figure \ref{fig:spp_cg}.
Firstly, we apply layer normalization \citep{ba2016layer} to all pattern scores
without any additional affine transformations. We omit the affine transformation
to not alter any of the pattern score information content and use layer
normalization as an expedient means of projecting the values of pattern scores
to a standard normal distribution. This normalization process guarantees that
pattern scores would be small in size and have a roughly even distribution of
positive and negative values around 0.

Layer normalization becomes very useful as we correspondingly encounter the
TauSTE layer. Here, the TauSTE layer maps all inputs which are strictly larger
than the $\tau \in \mathbb{R}$ threshsold to 1 and all others inputs to zero.
Naturally, the TauSTE layer is only useful when it is able to discriminate the
inputs by mapping some of them to 1 and some to 0, instead of always mapping all
inputs to either 1 or 0. Without layer normalization, the TauSTE layer would not
be able to perform its function since pattern scores tend to be mostly positive
with differing ranges. Layer normalization therefore helps to project these
variations of pattern scores to a uniform range; which ultimately allows the TauSTE
layer to discriminate inputs and produce diverse binary outputs.

A natural criticism of the TauSTE layer could be that it strongly limits the
flow of information in the SoPa++ model. While the binarization present in this
layer does indeed limit the rich flow of continuous numerical information, it is
still worth noting that this layer can preserve significant information given a
sufficiently large $m$ value, corresponding to the number of strict linear-chain
WFA-$\omega$ and TauSTE neurons. For example, if we allow for $m=40$ and therefore provide 40
WFA-$\omega$ and TauSTE neurons, we can have a total of
2$^{40}\approx1.1\times10^{12}$ binary state possibilities; which is slightly
greater than one trillion possible binary states. To provide some context to this
order of magnitude, the aforementioned number of possibilities is roughly equal
to estimated number of stars present in the Andromeda Galaxy
\citep{10.1093/mnras/stu879}. This would imply that despite the reduction in
information content on the TauSTE layer, there are still sufficient mappable
states available to learn various representations relevant to output classes;
given a large enough $m$ value or number of WFA-$\omega$.

After binarizing the hidden values in the TauSTE layer, we apply a simple linear
transformation to the TauSTE binary outputs to modify their dimensionality from
$m$ to $n \in \mathbb{N}$, where the $n$ represents the number of output
classes. We specifically chose a linear transformation layer over a MLP because
linear regressors are known to be transparent models
\citep{arrieta2020explainable} and this is a feature which ultimately assists us
in our explanations by simplification post-hoc explainability technique, which
we describe in greater detail in the next sections. Finally, we apply a softmax
function over the linear outputs to project them to probabalistic spaces and
then compute an argmax to extract the highest scoring index which represents the
predicted class. In the case of Figure \ref{fig:spp_cg}, the output class for
the input pre-processed utterance \texttt{``[START] 10 day weather forecast
  [END]''} is the \texttt{weather/find} class which corresponds to class index 11.

\subsection{Transparency}

\label{section:spp_transparency}

In regards to transparency, SoPa++ is a hybridized model consisting of RNN, CNN
and weighted finite-state neural components similar to that of SoPa. Following the
arguments of \citet{arrieta2020explainable} related to the black-box natures of
RNNs and CNNs, we can conclude that SoPa++ would correspondingly fall into the
black-box model category. Because of this classification, the SoPa++ model
would require post-hoc explainability techniques in order to explain its inner
mechanisms.

Reviewing the post-hoc explainability techniques offered by SoPa as per Section
\ref{section:sopa_post_hoc}, we would opine that a major limitation of these
techniques is that they do not fully capitalize on the rich theoretical
foundations offered by WFAs; such as their possible conversions to NFA and
ultimately regular expressions. In order to address this limitation, we propose
a new explanations by simplification post-hoc explainability technique for
simplifying a fully trained black-box SoPa++ model into a transparent regular
expression proxy model. We describe this new explanations by simplification
post-hoc explainability technique in the next section.

\section{RE proxy}

In this section, we describe the key processes required to convert a
fully trained black-box SoPa++ model into a transparent RE proxy model. These include
the introduction of a path-augmented document scoring algorithm and the creation
of the RE lookup layer. Next, we describe the computational graph or forward
pass of the RE proxy model. Finally, we provide some comments on
explainability-related aspects of the simplified RE proxy model and its
antecedent SoPa++ counterpart.

\subsection{Path-augmented document score}

\begin{algorithm}[t!]
  \small
  \caption{Strict linear-chain WFA-$\omega$ path-augmented document score$^*$}
  \label{algo:lc_wfa_w_document_score_trace}
  \begin{algorithmic}[1]
    \Require{Strict linear-chain WFA-$\omega$ (denoted as $\mathcal{A}$) and document $\bm{y}$}
    \Ensure{Document score $s_{\text{doc}}(\bm{y})$ and its corresponding path $\pi_{\text{doc}}(\bm{y})$}
    \Statex
    \Function{docscore\_path}{$\mathcal{A}, \bm{y}$}
    \State $\bm{h}_0 \gets \big[\bar{1}, -\infty, \ldots, -\infty\big]$ \Comment{Create
      initial hidden state vector $\bm{h}_0: \mathcal{Q} \rightarrow \mathbb{K}$}
    \For{$i \gets 1,2,\ldots,\bm{|y|}$} \Comment{Sequentially iterate over each
      token $y_i \in \bm{y}$}
    \State $\bm{m} \gets \big[[\bm{\Gamma}(y_i)]_{1,2}, [\bm{\Gamma}(y_i)]_{2,3}, \ldots,
    [\bm{\Gamma}(y_i)]_{|\mathcal{Q}|-1,|\mathcal{Q}|}\big]$ \Comment{Main-path scores for token $y_i$}
    \State $\bm{\omega} \gets \big[[\bm{\Gamma}(\omega)]_{1,2}, [\bm{\Gamma}(\omega)]_{2,3}, \ldots,
    [\bm{\Gamma}(\omega)]_{|\mathcal{Q}|-1,|\mathcal{Q}|}\big]$
    \Comment{State-wise wildcard scores}
    \State $\bm{m'} \gets \bm{m} \otimes \bm{h}_{i-1}[:-1]$ \Comment{Path
      score with main-path transitions$^{\dagger}$}
    \State $\bm{\omega'} \gets \bm{\omega} \otimes \bm{h}_{i-1}[:-1]$ \Comment{Path
    score with $\omega$ transitions$^{\dagger}$}
    \State $\bm{h}_{i} \gets [\bar{1}] \mathbin\Vert \max(\bm{m'}, \bm{w'})$
    \Comment{Concatenate $\bar{1}$ to maximum of $\bm{m'}$ and $\bm{\omega'}$}
    \State $\pi_i \gets \text{trace}(h_i[-1])$
    \Comment{Back-trace path $\pi_i$ corresponding to $h_i[-1]^{\ddagger}$}
    \EndFor
    \State $j \gets \argmax_{i \in 1,2,...,|\bm{y}|}
    \bm{h}_{i}[-1]$
    \Comment{Get index of final states' maximum$^{\ddagger}$}
    \State $s_{\text{doc}}(\bm{y}) \gets \bm{h}_{j}[-1]$
    \Comment{Get maximum final state value$^{\ddagger}$}
    \State $\pi_{\text{doc}}(\bm{y}) \gets \pi_j$
    \Comment{Get path of maximum final state value}
    \State \Return $[s_{\text{doc}}(\bm{y}), \pi_{\text{doc}}(\bm{y})]$
    \EndFunction
  \end{algorithmic}
  \algcomment{
    $^*$$\otimes$ and $\bar{1}$ are
    derived from max-based semirings where all semiring operations are element-wise \\
    $^{\dagger}\bm{h}_{i}[:-1]$ follows the Python indexing syntax
    and implies keeping all elements of $\bm{h}_{i}$ except the last \\
    $^{\ddagger}\bm{h}_{i}[-1]$ follows the Python indexing syntax and implies
    retrieving the last element of the vector $\bm{h}_{i}$
  }
\end{algorithm}

One major advantage of the Viterbi algorithm, as per Definition
\ref{def:string_score}, is its ability to return the highest path score which
can ultimately allow for attribution to a certain path and substring in a
document. In order to simplify SoPa++ into a RE proxy model, we first need to
modify our document scoring algorithm to not only return the document score
$s_{\text{doc}}(\bm{y})$ but also its corresponding path
$\pi_{\text{doc}}(\bm{y})$. This process is described as a path-augmented
document score in Algorithm \ref{algo:lc_wfa_w_document_score_trace}, where we
trace the exact path of each transition and return the best path in addition to
the highest path score. Similar to Algorithm \ref{algo:lc_wfa_w_document_score},
this algorithm has a time-complexity of $O(|Q||\bm{x}|)$, where
$|\mathcal{Q}|$ refers to the number of states in a strict linear-chain
WFA-$\omega$ and $|\bm{x}|$ refers to the length of the input string. It is also
worth mentioning that the highest scoring path returned ultimately reflects a set of
transitions in the form of a strict linear-chain NFA; which can correspondingly
be transformed into a regular expression. Therefore, it can be inferred that
Algorithm \ref{algo:lc_wfa_w_document_score_trace} returns the best scoring
regular expression corresponding to a certain strict linear-chain WFA-$\omega$.

\subsection{RE lookup layer}

The next step in simplifying a fully-trained SoPa++ to a RE proxy model is the
extraction of a RE lookup layer. To motivate this process, we shortly revert to
the SoPa++ computational graph in Figure \ref{fig:spp_cg}. The TauSTE layer
filters input signals from normalized pattern scores and provides +1 outputs for
normalized pattern scores which exceed the $\tau$-threshold. Since pattern
scores correspond to document scores and document scores can be augmented with
best paths and regular expressions as per Algorithm
\ref{algo:lc_wfa_w_document_score_trace}, we can assign the activation of each
TauSTE neuron for each input utterance with an "activating" regular expression.
By iterating over all our training data, we can collect many such "activating"
regular expressions which activate the various TauSTE neurons; such that each
TauSTE neuron $\textbf{N}_i$ is assigned a set of "activating" regular expressions
\textbf{\{RE\}$_i$}. The collection of all regular expressions
$[\{\textbf{RE}\}_1, \ldots, \{\textbf{RE}\}_m]$ which activate $m$ TauSTE
neurons is known as the RE lookup layer. Overall, the RE lookup layer represents
a knowledge base of important regular expressions that cause the SoPa++ model to
make weighted decisions. This process of extracting the RE lookup layer from
the SoPa++ model is reflected in Algorithm \ref{algo:simplification_process}.

\begin{algorithm}[t!]
  \small
  \caption{Extracting RE lookup layer from SoPa++}
  \label{algo:simplification_process}
  \begin{algorithmic}[1]
    \Require{Trained SoPa++ model $\mathcal{S}$ and training data $[\bm{y}_1, \ldots, \bm{y}_t]$}
    \Ensure{RE lookup layer $[\{\textbf{RE}\}_1, \ldots, \{\textbf{RE}\}_m]$}
    \Statex
    \Function{extract\_lookup}{$\mathcal{S}, [\bm{y}_1, \ldots, \bm{y}_t]$}
    \State $[\{\textbf{RE}\}_1, \ldots, \{\textbf{RE}\}_m] \gets [\emptyset,
    \ldots, \emptyset]$
    \Comment{Initialize RE lookup layer}
    \For{$i \gets 1,2,\ldots,t$}
    \Comment{Loop over training data}
    \For{$j \gets 1,2,\ldots,m$}
    \Comment{Loop over WFA-$\omega$'s in $\mathcal{S}$}
    \State $\mathcal{A}_j \gets \text{get\_WFA}(\mathcal{S}, j)$
    \Comment{Get WFA-$\omega$ by index}
    \State $s^j_{\text{doc}}(\bm{y}_i), \pi^j_{\text{doc}}(\bm{y}_i) \gets
    {\small\textsc{docscore\_path}}(\mathcal{A}_j, \bm{y}_i)$
    \Comment{As per Algorithm \ref{algo:lc_wfa_w_document_score_trace}}
    \If{TauSTE$(s^j_{\text{doc}}(\bm{y}_i)) == 1$}
    \State $\{\textbf{RE}\}_j \gets \{\textbf{RE}\}_j \cup
    \pi^j_{\text{doc}}(\bm{y}_i)$
    \Comment{Save $\pi^j_{\text{doc}}(\bm{y}_i)$ if it activates TauSTE} 
    \EndIf
    \EndFor
    \EndFor
    \State \Return $\text{compress}([\{\textbf{RE}\}_1, \ldots, \{\textbf{RE}\}_m])$
    \Comment{Compress RE lookup layer} 
    \EndFunction
  \end{algorithmic}
\end{algorithm}

This leads us to several interesting conclusions. Firstly, we observe here that
the TauSTE layer is not only useful for reducing information complexity; but
also for attributing causal links to SoPa++'s decision-making. In this case,
the causal links are the "activating" regular expressions returned by the
strict-linear chain WFA-$\omega$ when computing the path-augmented document
score. Next, we can also observe the effect of the $\tau$-threshold on the RE
lookup layer. If the $\tau$-threshold is low, we effectively allow more TauSTE
neurons to be activated and therefore allow more regular expressions from the
strict linear-chain WFA-$\omega$ to be saved in the RE lookup layer.
Contrastingly, if the $\tau$-threshold is high; we allow fewer TauSTE neurons to
be activated and therefore allow fewer regular expressions to be saved in the RE
lookup layer. It is not necessarily clear whether small or large
$\tau$-thresholds are better for performance; but a larger $\tau$-threshold and
therefore smaller RE lookup layer could be beneficial for explainability
purposes since the RE knowledge base would be smaller and possbily easier
to comprehend for a human. 

\begin{figure}[t!]
  \centering
  \includegraphics[width=15cm]{pdfs/generated/regex_computational_graph/main.pdf}
  \caption{Visualization of the RE proxy computational graph}
  \label{fig:regex_cg}
\end{figure}

\subsection{Assembling RE proxy}

The final step in simplifying a fully-trained SoPa++ model into a RE proxy model
is to assemble the RE proxy model using the RE lookup and SoPa++ linear layers.
Specifically, for a given SoPa++ model; we firstly extract the RE lookup layer.
Then, we combine the RE lookup layer and the SoPa++ linear layer in order to
create the RE proxy model. A visualization of this is shown in Figure
\ref{fig:regex_cg} and we can observe how the RE lookup layer essentially
replaces most of the lower neural components of the SoPa++ model up until the
TauSTE layer. The resulting RE proxy should ideally be a good approximator of
the SoPa++ model; with the exact degree of approximation being reserved for
empirical investigation. Given this process of assembly, it is important to note
that each SoPa++ model allows for exactly one RE proxy model that can be
assembled. Therefore, SoPa++ and RE proxy models occur in model-pairs.

One major limitation of the RE proxy model is in its RE lookup layer. Since the
RE lookup layer is essentially a memorized knowledge base, it could contribute
to overfitting on the training data and result in a RE proxy model that is not
representative of the SoPa++ model on unseen data. While this limitation could
theoretically be offset by the presence of sufficient variable-length regular
expression samples with wildcards, it would ultimately boil down to an empirical
investigation to quantify how similarly the RE proxy model performs compared to
its antecedent SoPa++ model on previously unseen data.

\subsection{Computational graph}

\label{section:re_cg}

We now provide a description of the computational graph or forward pass of the
assembled RE proxy model with reference to Figure \ref{fig:regex_cg}. Similar to
the computational graph of SoPa++, we process our input text sequentially.
However, instead of using token embeddings and neural network components; we
pass our input utterance through our RE lookup layer which conducts substring
matches over all sets of regular expressions in $[\{\textbf{RE}\}_1, \ldots,
\{\textbf{RE}\}_m]$. If any regular expression set $\{\textbf{RE}\}_i$ matches
the input utterance, the index of this RE set if given a value of 1. If no
regular expression in the set matches, the index of this RE set is given a value
of 0. These values are assembled into a binary vector that mimics the outputs of
the TauSTE layer in the antecedent SoPa++ model. This binary vector is then
passed to the linear regression layer; which transforms the binary vector back
into continuous space with the appropriate dimensionality. Following softmax and
argmax operations, we arrive at our predicted output class. It is worth
mentioning that while the execution of the SoPa++ forward pass can be
accelerated due to tensor-based parallelization with hardware acceleration on a
Graphics Processing Unit (GPU), the forward pass of the RE proxy model is
significantly slower since we utilized an unoptimized single-threaded lookup
process to match regular expressions in the RE lookup layer.

\subsection{Transparency}

\label{section:re_transparency}

So far, we have described the process of simplifying a fully-trained black-box
SoPa++ model into a RE proxy model. We now investigate the RE proxy model and
comment on its degree of transparency. In essence, a RE proxy model consists of
a RE lookup layer followed by a linear regressor. The RE lookup layer can be
seen as a rules-based learner component where inputs are processed via clear and
interpretable RE matching rules to produce outputs. Since both rules-based
learners and linear regressors can be considered as transparent models as per
\citet[Page 7, Section 3]{arrieta2020explainable}, we could theoretically
classify the RE proxy model as a transparent model. However, it is important to
note that this is only a theoretical argument which could be disproven in given
practical cases. For example as per \citet[Page 9, Table
2]{arrieta2020explainable}, rules-based learners and linear regressors could be
viewed as black-box models if they require the handling of an incomprehensible
number of rules or input features. This could also be the case for the RE proxy
model depending most importantly on the size of the RE lookup layer.

\subsection{Explainability}

We now provide additional comments on the explanations by simplification
post-hoc explainability technique to simplify a fully-trained black-box SoPa++
model into a transparent RE proxy model. Firstly, we would assign the target
audience of this explainability technique as expert-users compared to the
inferred target audience of average end-users for the SoPa model. We designate
expert-users as our target audience mainly because the process of constructing
and understanding the RE proxy model is not as straightforward as the local
explanations and feature relevance techniques in SoPa as per Section
\ref{section:sopa_post_hoc}.

Next, we evaluate the quality of explanations offered by using the explanations
by simplification technique using the three guidelines offered in Section
\ref{section:xai_metrics}. Regarding the constrictive quality, it is likely that
the RE proxy model meets this criterion since the linear layer contains easily
interpretable weights which could explain which matched regular expressions were
scored higher or lower. Regarding causal links, it is likely that the RE proxy
model meets this criterion since the RE lookup layer matches fixed REs whose
binary matching scores are then clearly propagated from the start to the end of
the model. Therefore, a decision occurring at the end of the model could be
causally attributed to features at the start of the model. Regarding the selective
quality, it is possible that the RE proxy model meets this criterion since the
linear weights applied to matched regular expressions can be easily ranked to
understand which were the most important causal links influencing the model's
decision.

\begin{table}[t!]
  \centering \def\arraystretch{1.3}
  \begin{tabular}{L{0.275\linewidth} L{0.3\linewidth} L{0.3\linewidth}}
    \toprule
    Characteristic & SoPa & SoPa++ \\
    \midrule
    Text casing & True-cased & Lower-cased \\ 
    Token embeddings & GloVe 840B 300-dimensions & GloVe 6B 300-dimensions \\
    WFAs & Linear-chain WFAs with $\epsilon$, self-loop and main-path transitions & Strict linear-chain WFA-$\omega$'s with $\omega$ and main-path transitions \\
    Hidden layers & Multi-layer perceptron after max-pooling & Layer normalization, TauSTE and linear transformation after max-pooling \\
    Post-hoc explainability technique(s) & Local explanations, feature relevance & Explanations by simplification \\
    \bottomrule
  \end{tabular}
  \caption{Summarized differences for SoPa vs. SoPa++}
  \label{tab:sopa_spp_comparison}
\end{table}

\section{Differences between SoPa and SoPa++}

To wrap up this current segment on the SoPa++ model, we summarize the key
differences between SoPa and SoPa++ as shown in Table
\ref{tab:sopa_spp_comparison}. The most significant changes in SoPa++
include the modification of linear-chain WFAs to strict linear-chain
WFA-$\omega$'s, the replacement of the MLP with layer normalization, TauSTE and
linear layers and the introduction of the explanations by simplification
post-hoc explainability technique to create RE proxy models. With the
construction of SoPa++ and its RE proxy models established, we now proceed to
describe the methodologies used to answer our three research questions.

\section{RQ1: Evaluating performance of SoPa++}

In this section, we describe the methodologies used to answer our first research
question regarding the competitive performance of the SoPa++ model on the FMTOD
data set. Specifically, we describe how we trained the SoPa++ model and
afterwards, how we proceeded to evaluate and compare its performance to other
studies.

\subsection{Training}

\label{section:spp_training}

First and foremost, we address the issue of data imbalance in the FMTOD data set
as mentioned in Section \ref{section:fmtod_summary}. For this, we chose a simple
but effective solution of upsampling all minority data classes such that they
would have the same frequency as the majority class. We chose this approach over
other approaches such as gradient-weighting since this is generally a model
agnostic and straightforward approach.

Returning to the computational graph of SoPa++ in Figure \ref{fig:spp_cg}, we
compute the cross-entropy loss using the softmax output and target one-hot
encoded classes. Since SoPa++ is in essence a deep neural network, we utilize
the well-studied gradient descent technique to train and update our SoPa++ model
with the objective of minimizing the aforementioned cross-entropy loss. To
facilitate this learning process, we utilize
PyTorch\footnote{https://pytorch.org/} to assist with gradient computations,
backward passes and parallelized tensor computations. We utilize the Adam
optimizer \citep{DBLP:journals/corr/KingmaB14} to stabilize the gradient descent
learning technique with $\beta_1=0.9$ and $\beta_2=0.999$. In terms of fixed
training hyperparameters, we utilize a learning rate of 0.001, a batch size of
256 input utterances and neuron/word dropout probabilities of 0.2. We only apply
neuron dropouts on the transition matrices of all the strict linear-chain
WFA-$\omega$'s in SoPa++. We furthermore apply batch sorting based on input
utterance lengths to ensure maximum efficiency when computing pattern/document
scores. Based on our own experiments, we observed more stable training with the
max-sum semiring compared to the max-product semiring. For simplicity, we
therefore only use the max-sum semiring in all of our WFA-$\omega$'s.

\begin{table}[t!]
  \centering
  \begin{tabular}{lll}
    \toprule
    Model size & Pattern hyperparameter $P$ & Parameter count \\
    \midrule
    Small & \texttt{6-10\_5-10\_4-10\_3-10} & 1,260,292 \\
    Medium & \texttt{6-25\_5-25\_4-25\_3-25} & 1,351,612  \\
    Large & \texttt{6-50\_5-50\_4-50\_3-50} & 1,503,812 \\
    \bottomrule
  \end{tabular}
  \caption{Three different SoPa++ model sizes used during training with
    corresponding pattern hyperparameter $P$ and parameter counts}
  \label{tab:model_types}
\end{table}

To obtain some variation in our SoPa++ models, we decide to use a grid-search
technique while varying the patterns $P$ and $\tau$-threshold hyperparameters.
Our variations in the patterns hypereparameter $P$ result in three different
SoPa++ model sizes which we define as small, medium and large as per Table
\ref{tab:model_types}. Correspondingly, we vary the $\tau$-threshold
hyperparameter with the following five possible values: $\{0.00, 0.25, 0.50,
0.75, 1.00\}$. For each model configuration in our grid-search routine, we
repeat a model run ten times with different initial random seeds in order to
obtain a distribution of performances. In total, our grid-search routine
produces a total of $3\times5\times10=150$ model runs. Finally, we train all
models for a maximum of 50 epochs with 10 patience epochs for early stopping in
case of a performance plateau or worsening. To trigger early stopping in the
patience framework, we monitor the cross-entropy loss over the validation data
set. We run all SoPa++ training experiments on a single NVIDIA GeForce GTX 1080
Ti GPU for $\sim$24 hours.

\subsection{Evaluation}

Given fully trained SoPa++ models from the previous training step, we now
proceed to evaluate the performance of the SoPa++ models on the FMTOD data set's
test partition. We simply run the preprocessed FMTOD test partition through the
computational graph of the SoPa++ model and obtain the predicted classes. With
the predicted and target classes, we compute the accuracy of the SoPa++ models
and summarize these to obtain a distribution of performances over the random
seed iterations. Finally, we compare our mean accuracies with the accuracy range
of other recent papers on FMTOD as described in Section
\ref{section:fmtod_performance}. If the mean accuracy of our SoPa++ models
falls into the aforementioned competitive range, we can then conclude that our
SoPa++ model performs competitively with other recent studies.

\section{RQ2: Evaluating explanations by simplification}

\label{section:evaluate_explain}

We now move on to describe the methodologies pursued to answer our second
research question on evaluating the effectiveness of our explanations by
simplification post-hoc explainability method. As mentioned in Definition
\ref{def:explain_simplify}, the purpose of explanations by simplification is to
create a less complex proxy model which can both keep a similar performance
score and maximize its resemblance to the antecedent model. We already discussed
how the RE proxy derived from SoPa++ is likely to be a transparent model
compared to the black-box SoPa++ model; which already satisfies the first
criterion that the proxy model should be less complex than the antecedent model.

To address the next criterion regarding the similarity of the performance scores
of both SoPa++ and RE proxy model pairs, we compute and compare the accuracy
scores of all model pairs on the FMTOD data set's test partition since this
represents previously unseen data for both SoPa++ and RE proxy models. To
address the final criterion of maximum resemblance between the antecedent and
proxy models, we propose computing the softmax distance norm and binary
misalignment rate distance metrics to quantify the distance between the SoPa++
and RE proxy model pairs. Naturally, a smaller distance metric would symbolize
greater resemblance between model pairs. We describe these distance metrics with
mathematical formalisms in the following sections.

\subsection{Softmax distance norm}

The softmax distance norm $\delta_{\sigma}(\bm{y})$ refers to the Euclidean norm
of the difference in the softmax vectors of the SoPa++ and RE proxy models for a
given document $\bm{y}$, which are represented by
$\bm{\sigma_{\mathcal{S}}}(\bm{y})$ and $\bm{\sigma_{\mathcal{R}}}(\bm{y})$
respectively:

\begin{equation}
  \delta_{\sigma}(\bm{y}) = \left\Vert \bm{\sigma_{\mathcal{S}}}(\bm{y}) - \bm{\sigma_{\mathcal{R}}}(\bm{y}) \right\Vert_{2} = \sqrt{\sum^n_{i=1} (\sigma_{\mathcal{S}_i}(\bm{y}) - \sigma_{\mathcal{R}_i}(\bm{y}))^2} 
\end{equation}

In order to measure the central tendency of the softmax distance norm across the
FMTOD data set's test partition, we compute this distance metric for all
instances and correspondingly compute the mean softmax distance norm; which we
denote here as $\overline{\delta_{\sigma}}$. A low mean softmax distance norm
norm would imply that the softmax distributions of SoPa++ and RE proxy models
were similar and would imply, to a certain extent, that the models conducted
similar weightings of input features. This method is not a perfect indicator of
interpreting distances between models; but it is more refined than analyzing
discrete classification outputs.

\subsection{Binary misalignment rate}

The binary misalignment rate $\delta_b(\bm{y})$ refers to the dimension
normalized Manhattan norm of the difference in the binary TauSTE vectors of the
SoPa++ and RE proxy models for a given document $\bm{y}$, which are represented
by $\bm{b_{\mathcal{S}}}(\bm{y})$ and $\bm{b_{\mathcal{R}}}(\bm{y})$
respectively:

\begin{equation}
  \delta_b(\bm{y}) = \dfrac{\left\Vert \bm{b_{\mathcal{S}}}(\bm{y}) - \bm{b_{\mathcal{R}}}(\bm{y}) \right\Vert_{1}}{dim(\bm{b_{\mathcal{S}}}(\bm{y}) - \bm{b_{\mathcal{R}}}(\bm{y}))} = \dfrac{\sum^n_{i=1} |b_{\mathcal{S}_i}(\bm{y}) - b_{\mathcal{R}_i}(\bm{y})|}{{dim(\bm{b_{\mathcal{S}}}(\bm{y}) - \bm{b_{\mathcal{R}}}(\bm{y}))}}
\end{equation}

Here, the $dim$ operator retrieves the dimensionality of a vector space.
Normalization of the Manhattan norm with this dimensionality is necessary since
we compare models with different sizes which could have different binary vector
dimensions. In order to measure the central tendency of the binary misalignment
rate across the FMTOD data set's test partition, we compute this distance metric
for all instances and correspondingly compute the mean binary misalignment rate;
which we denote here as $\overline{\delta_b}$. A low mean binary misalignment
rate would imply that the TauSTE binary vector distributions were close
together. Therefore, the binary misalignment could be used as another indicator
to measure the distance between SoPa++ and RE proxy model pairs.

\section{RQ3: Interesting and relevant explanations}

Finally, we arrive at the methodologies used to answer our third research
question related to showing interesting and relevant explanations that the
SoPa++ and RE proxy models can provide on the FMTOD data set. To approach this
research question, we pursue two key strategies of visualizing relative linear
weights and sampling REs from the RE lookup layer corresponding to salient
TauSTE neurons.

\subsection{Relative linear weights}

A major advantage of the linear layer in both SoPa++ and RE proxy models is its
interpretability or transparency compared to the MLP in the SoPa model. To
visualize the relative linear weights applied to the TauSTE neurons, we apply a
softmax operation over weights in the linear layer assigned to each TauSTE
neuron and visualize these relative weights to observe how SoPa++ and its RE
proxy models distribute feature importance across TauSTE neurons.

\subsection{Samples from RE lookup layer}

Based on the aforementioned visualization of relative linear weights, we
identify salient TauSTE neurons which receive disproportionately large relative
linear weights for the alarm, reminder and weather domains. Correspondingly,
we probe the RE lookup layer corresponding to these salient TauSTE neurons and
sample ten "activating" regular expressions for each of these salient TauSTE
neurons. Theoretically, this could provide us with an insight as to which
regular expressions were of particular importance for the classification of the
three domains. Furthermore, a deeper analysis of these regular expressions
could provide us with insights into possible inductive biases incorporated by the
SoPa++ and the RE proxy models.

%%% Local Variables: 
%%% mode: latex
%%% TeX-master: "main"
%%% End: 
\chapter{Results}

\label{chapter:results}

In this chapter, we focus on summarizing the results of our aforementioned
methodologies. As a note, we do not dive deep into answering our research
questions here. Instead, we simply use this chapter to report on our results and
we answer our research questions in the next chapter.

\section{RQ1: Evaluating performance of SoPa++}

Following our methodologies, we conducted a grid-search training paradigm where
we varied our patterns and $\tau$-threshold parameters to obtain a total of 15
different modelling runs. Given that we repeated unique model run 10 times, we
ultimately ran our grid-search for 150 model runs. This process took roughly 24
hours on a single NVIDIA GeForce GTX 1080 Ti GPU running light, medium and heavy
model runs concurrently.

\subsection{Training}

We first describe the results of our training. Figure \ref{fig:results_training}
shows the progress of the validation accuracy against training updates. The
different coloured lines indicate the various initial random seeds assigned to
the particulat model run. We can observe that increasing the $\tau$ value from 0
to 1 tends to decrease the overall validation accuracy profile. Correspondingly,
we can also observe that the larger model tends to have a higher validation
accuracy profile compared to the smaller models. We refer to the pattern
hyperparameter as $P$ in Figure \ref{fig:results_training}. Finally, we can
observe that the larger models tend to have an earlier convergence or
early-stopping window compared to the smaller models.

\subsection{Evaluation}

In regards to the evaluation of SoPa++ performance, we can refer to Table
\ref{tab:results_evaluation} for a tabular summary of test accuracies across the
light, medium and heavy model variants grouped by the various $\tau$-thresholds.
The exact specification of the light, medium and heavy model variants were
described in Section \ref{section:spp_training}. Here, we can observe that the
best performing models were the heavy models with $\tau$=0.0 and $\tau$=0.25
and the medium model with $\tau$=0.0. We can observe that test accuracies
generally show a decreasing trend as we increase the $\tau$-threshold.
Correspondingly, we can observe a decreasing performance trend as we decrease
the size of the model from heavy to light models. We can also observe that the
standard deviations in performance are generally similar.

\begin{figure}[t!]
  \centering
  \includegraphics[width=14cm]{pdfs/generated/train_spp_grid_1617362157.pdf}
  \caption{Visualization of validation accuracy against number of training
    updates for grid-search grouped by pattern hyperparameters and $\tau$-thresholds}
  \label{fig:results_training}
\end{figure}

\begin{table}[t!]
  \centering \def\arraystretch{1.3}
  \small
  \begin{tabular}{lllllll}
    \toprule
    && \multicolumn{5}{c}{Accuarcy in $\%$ with mean $\pm$ standard-deviation} \\
    \cline{3-7} \\[-10pt]
    Model & Parameters & $\tau$=0.0 & $\tau$=0.25 & $\tau$=0.5 & $\tau$=0.75 & $\tau$=1.0 \\
    \midrule
    Light & 1,260,292 & 97.6 $\pm$ 0.2 & 97.6 $\pm$ 0.2 & 97.3 $\pm$ 0.2 & 97.0 $\pm$ 0.3 & 96.9 $\pm$ 0.3 \\
    Medium & 1,351,612 & \textbf{98.3 $\bm{\pm}$ 0.2} & 98.1 $\pm$ 0.1 & 98.0 $\pm$ 0.2 & 97.9 $\pm$ 0.1 & 97.7 $\pm$ 0.1  \\
    Heavy & 1,503,812 & \textbf{98.3 $\bm{\pm}$ 0.2} & \textbf{98.3 $\bm{\pm}$ 0.2} & 98.2 $\pm$ 0.2 & 98.1 $\pm$ 0.2 & 98.0 $\pm$ 0.2 \\
    \bottomrule
  \end{tabular}
  \caption{Test accuracies of the SoPa++ models grouped by model sizes and
    $\tau$-thresholds; accuracies and standard deviations were calculated across
  random seed iterations}
  \label{tab:results_evaluation}
\end{table}

\section{RQ2: Evaluating explanations by simplification}

% > aggregate(regex_acc ~ patterns + tau_threshold, data=collections, FUN=mean)
%               patterns tau_threshold regex_acc
% 1  6-10_5-10_4-10_3-10             0 0.9549596
% 2  6-25_5-25_4-25_3-25             0 0.9745687
% 3  6-50_5-50_4-50_3-50             0 0.9789218
% 4  6-10_5-10_4-10_3-10          0.25 0.9617385
% 5  6-25_5-25_4-25_3-25          0.25 0.9748113
% 6  6-50_5-50_4-50_3-50          0.25 0.9803639
% 7  6-10_5-10_4-10_3-10           0.5 0.9645957
% 8  6-25_5-25_4-25_3-25           0.5 0.9747305
% 9  6-50_5-50_4-50_3-50           0.5 0.9803369
% 10 6-10_5-10_4-10_3-10          0.75 0.9634906
% 11 6-25_5-25_4-25_3-25          0.75 0.9752022
% 12 6-50_5-50_4-50_3-50          0.75 0.9803100
% 13 6-10_5-10_4-10_3-10             1 0.9610243
% 14 6-25_5-25_4-25_3-25             1 0.9748518
% 15 6-50_5-50_4-50_3-50             1 0.9805256
% > aggregate(regex_acc ~ patterns + tau_threshold, data=collections, FUN=sd)
%               patterns tau_threshold   regex_acc
% 1  6-10_5-10_4-10_3-10             0 0.009687851
% 2  6-25_5-25_4-25_3-25             0 0.005038281
% 3  6-50_5-50_4-50_3-50             0 0.004365320
% 4  6-10_5-10_4-10_3-10          0.25 0.008446945
% 5  6-25_5-25_4-25_3-25          0.25 0.003450913
% 6  6-50_5-50_4-50_3-50          0.25 0.003337103
% 7  6-10_5-10_4-10_3-10           0.5 0.005537846
% 8  6-25_5-25_4-25_3-25           0.5 0.002384975
% 9  6-50_5-50_4-50_3-50           0.5 0.002853588
% 10 6-10_5-10_4-10_3-10          0.75 0.006689082
% 11 6-25_5-25_4-25_3-25          0.75 0.002635605
% 12 6-50_5-50_4-50_3-50          0.75 0.002482981
% 13 6-10_5-10_4-10_3-10             1 0.006328909
% 14 6-25_5-25_4-25_3-25             1 0.003111233
% 15 6-50_5-50_4-50_3-50             1 0.002347448

% > aggregate(softmax_distance ~ patterns + tau_threshold, data=collections, FUN=mean)
%               patterns tau_threshold softmax_distance
% 1  6-10_5-10_4-10_3-10             0       0.15010717
% 2  6-25_5-25_4-25_3-25             0       0.09476992
% 3  6-50_5-50_4-50_3-50             0       0.07106383
% 4  6-10_5-10_4-10_3-10          0.25       0.11310785
% 5  6-25_5-25_4-25_3-25          0.25       0.07315419
% 6  6-50_5-50_4-50_3-50          0.25       0.04993503
% 7  6-10_5-10_4-10_3-10           0.5       0.09955704
% 8  6-25_5-25_4-25_3-25           0.5       0.06148820
% 9  6-50_5-50_4-50_3-50           0.5       0.04293018
% 10 6-10_5-10_4-10_3-10          0.75       0.10031126
% 11 6-25_5-25_4-25_3-25          0.75       0.05843249
% 12 6-50_5-50_4-50_3-50          0.75       0.04462260
% 13 6-10_5-10_4-10_3-10             1       0.10272472
% 14 6-25_5-25_4-25_3-25             1       0.06345463
% 15 6-50_5-50_4-50_3-50             1       0.04730200
% > aggregate(softmax_distance ~ patterns + tau_threshold, data=collections, FUN=sd)
%               patterns tau_threshold softmax_distance
% 1  6-10_5-10_4-10_3-10             0      0.022698184
% 2  6-25_5-25_4-25_3-25             0      0.016573534
% 3  6-50_5-50_4-50_3-50             0      0.014613714
% 4  6-10_5-10_4-10_3-10          0.25      0.021730004
% 5  6-25_5-25_4-25_3-25          0.25      0.009497534
% 6  6-50_5-50_4-50_3-50          0.25      0.008448988
% 7  6-10_5-10_4-10_3-10           0.5      0.016047160
% 8  6-25_5-25_4-25_3-25           0.5      0.008094996
% 9  6-50_5-50_4-50_3-50           0.5      0.005564666
% 10 6-10_5-10_4-10_3-10          0.75      0.015798138
% 11 6-25_5-25_4-25_3-25          0.75      0.004955638
% 12 6-50_5-50_4-50_3-50          0.75      0.004833917
% 13 6-10_5-10_4-10_3-10             1      0.017653580
% 14 6-25_5-25_4-25_3-25             1      0.005008278
% 15 6-50_5-50_4-50_3-50             1      0.005459310

% > aggregate(binary_distance ~ patterns + tau_threshold, data=collections, FUN=mean)
%               patterns tau_threshold binary_distance
% 1  6-10_5-10_4-10_3-10             0       0.2007382
% 2  6-25_5-25_4-25_3-25             0       0.2165814
% 3  6-50_5-50_4-50_3-50             0       0.2269580
% 4  6-10_5-10_4-10_3-10          0.25       0.1890425
% 5  6-25_5-25_4-25_3-25          0.25       0.2103988
% 6  6-50_5-50_4-50_3-50          0.25       0.2201441
% 7  6-10_5-10_4-10_3-10           0.5       0.1770896
% 8  6-25_5-25_4-25_3-25           0.5       0.2028309
% 9  6-50_5-50_4-50_3-50           0.5       0.2058340
% 10 6-10_5-10_4-10_3-10          0.75       0.1545125
% 11 6-25_5-25_4-25_3-25          0.75       0.1721154
% 12 6-50_5-50_4-50_3-50          0.75       0.1796683
% 13 6-10_5-10_4-10_3-10             1       0.1244646
% 14 6-25_5-25_4-25_3-25             1       0.1348997
% 15 6-50_5-50_4-50_3-50             1       0.1423266
% > aggregate(binary_distance ~ patterns + tau_threshold, data=collections, FUN=sd)
%               patterns tau_threshold binary_distance
% 1  6-10_5-10_4-10_3-10             0     0.015272982
% 2  6-25_5-25_4-25_3-25             0     0.015466187
% 3  6-50_5-50_4-50_3-50             0     0.008555918
% 4  6-10_5-10_4-10_3-10          0.25     0.006679655
% 5  6-25_5-25_4-25_3-25          0.25     0.015912701
% 6  6-50_5-50_4-50_3-50          0.25     0.009791348
% 7  6-10_5-10_4-10_3-10           0.5     0.015313128
% 8  6-25_5-25_4-25_3-25           0.5     0.014773546
% 9  6-50_5-50_4-50_3-50           0.5     0.010081221
% 10 6-10_5-10_4-10_3-10          0.75     0.013177274
% 11 6-25_5-25_4-25_3-25          0.75     0.008723274
% 12 6-50_5-50_4-50_3-50          0.75     0.009173360
% 13 6-10_5-10_4-10_3-10             1     0.017598744
% 14 6-25_5-25_4-25_3-25             1     0.010098175
% 15 6-50_5-50_4-50_3-50             1     0.007099766

%%% Local Variables: 
%%% mode: latex
%%% TeX-master: "main"
%%% End: 
\chapter{Discussion}

In this chapter, we investigate the implications of the aforementioned results
based on our methodologies and attempt to interpret these results in order to
answer our research questions. Furthermore, we evaluate to what extent these
results can answer our research questions and what limitations our methodologies
pose in this regard.

\section{RQ1: Evaluating performance of SoPa++}

In order to answer our first research question on whether SoPa++ can deliver
competitive performance for the English language intent detection task on the
FMTOD data set, we must compare the general performance range of our SoPa++
model(s) against the results from other recent studies as mentioned in Section
\ref{section:fmtod_performance}. Referring to our accuracy ranges from Table
\ref{tab:results_evaluation}, we can observe the SoPa++ model shows a general
accuracy range from 96.9-98.3$\%$ for the aforementioned task. This falls into
the general performance range of 96.6-99.5$\%$ observed in other studies; albeit
in the lower end of the performance spectrum. We can therefore conclude that
SoPa++ offers competitive performance on the FMTOD's English language intent
detection task compared to other recent studies.

While SoPa++'s performance range being on the lower spectrum can be seen as
disadvantageous, it is worth noting that the models it is being compared against
are vastly different. For one, the BERT models shown in
\ref{tab:fmtod_results} had parameter counts ranging from $\sim$110-340 million
parameters \citep{devlin-etal-2019-bert}; which are $\sim$100-300 times larger
than our SoPa++ model. It is therefore also useful to take these differences
into account when evaluating and comparing model performances. Next, models from
\citet{zhang-etal-2020-intent} showed an exceptionally high accuracy of
99.5$\%$ because of pre-training on the external WikiHow data set for general
intent detection tasks. As a result, it might be difficult to compare our
results with that of \citet{zhang-etal-2020-intent} since they utilized external
specialized data to pre-train their models while we did not.

\section{RQ2: Evaluating explanations by simplification}

In order to answer our second research question on whether SoPa++ provides
effective explanations by simplification, we need to first interpret some of the
results on comparing SoPa++ and RE proxy model pairs. Firstly, we can compare
the accuracy scores of SoPa++ and RE proxy model pairs as shown in the top
portion of Table \ref{tab:explain_evaluate_performance}. Here, we can observe
that the medium and heavy models with $\tau$-threshold values of 0.75 and 1.00
result in SoPa++ and RE proxy model pairs with performance differences as low as
$\sim$1-2$\%$. Similarly, medium and heavy models with $\tau$-thresholds of
0.50 and 0.75 show $\delta_{\sigma}$ metrics in the range of 4.3-5.8$\%$. We
interpret these results to imply that SoPa++ can provide effective explanations
by simplification for models with more WFA-$\omega$'s and with $\tau$-thresholds
in range of 0.50-1.00. This is however not necessarily the case for smaller
models or models with low $\tau$-thresholds.

While our results show that conversion of SoPa++ to RE proxy models are
generally effective with a minimal loss in performance and high resemblance
given large models and high $\tau$-thresholds; there are still certain
limitations to our RE proxy models. While we did mention that the RE proxy
models should theoretically be transparent models in Section
\ref{sec:re_transparency}, this could very easily no longer be the case given
large RE proxy models with too many internal regular expressions. As an example,
some of the RE lookup layers in heavy RE proxy models contain tens to hundreds
of thousands of REs. Reading and understanding all of these internal REs might
not be a practical task for a human, which may essentially render the
theoretically transparent RE proxy model as a black-box model in a practical
sense. This is a natural trade-off that we observe here.

\section{RQ3: Interesting and relevant explanations}

\label{section:discussion_regex}

In this section, we interpret the results of probing into our best performing
light RE proxy model in order to gain interesting and relevant explanations
regarding the FMTOD English language intent detection data task. For one, we can
analyze the TauSTE neuron-based weight distributions as shown in Figure
\ref{fig:neuron_weights} and observe that weights are generally continously
distributed across all neurons; with some exceptions such as neurons 21, 27 and
32 where the weights are more skewed towards the alarm, reminder and weather
sub-classes respectively. This implies that despite the quantization applied in
the TauSTE layer, the SoPa++ and RE proxy models still distribute feature
importances across neurons in a connective sense; which also implies that the
purpose of each neuron in making any decision would be difficult to understand
since each neuron would have a mixed impact on different classes. This could
possibly be an impediment to explainability since clear causes and effects
between neurons and output classes would be hard to identify.

Next, we can observe the RE samples from the RE lookup layer pertaining to
neurons 21, 27 and 32 in Figures \ref{fig:regex_example_neuron_21},
\ref{fig:regex_example_neuron_27} and \ref{fig:regex_example_neuron_32}
respectively. Firstly, we can observe a clear stratification of the types of REs
captured between neurons 21, 27 and 32. As mentioned earlier, neurons 21, 27 and
32 specialize in alarm, reminder and weather sub-classes respectively. Similary,
the REs captured for each of these neurons tend to show high similarity for each
of these sub-classes. This could imply that certain neurons which place high
weights on a particular sub-class set tend to gather similar REs as well.

Next, we can observe clustering or branching of REs in the RE lookup layer. For
example, the top-most RE in Figure \ref{fig:regex_example_neuron_21} shows that
the third transition capturing many different words. This branching can also be
observed in many other RE examples. Finally, another interesting phenomenon is
that this branching of RE transitions tend to have semantic similarities as
well. For example using the aforementioned RE example, we can observe words such
as ``sunday'' and ``thursdays'' occur in the same transition and could therefore
imply that days of the week trigger a high score in this transition.

Lastly, we can also observe certain inductive biases that the RE proxy model
(and correspondingly the SoPa++ model) has acquired from the training data set.
For example, the fourth and eighth REs from the top in Figure
\ref{fig:regex_example_neuron_27} show transitions with the word ``repairman''
and ``girlfriend'' which led to high path scores. While these words were indeed
relevant in both the training and test data set, the utility of a different
gender such as ``repairwoman'' or ``boyfriend'' might not lead to the same model
decisions since these words could be either outside of the model's vocabulary or
unseen and therefore poorly weighted. A major advantage of the SoPa++ and its RE
proxy model is that the user can have direct access to these representative REs
in the RE lookup layer and can theoretically parse through these and adjust
these ``problematic'' REs to adjust the inductive bias. One way to adjust the
inductive biases in this case could be to add branching transitions with the
different gender roles in the above single-gender cases. This is an advantageous
component of the SoPa++ model framework compared to other deep learning models
without explanations by simplification capabilities, since it would be difficult
if not impossible to easily discover such inductive biases within these
black-box models. 

%%% Local Variables: 
%%% mode: latex
%%% TeX-master: "main"
%%% End: 
\chapter{Conclusions}

\label{chapter:conclusions}

In this chapter, we summarize the key findings of this thesis. We
started off this thesis by emphasizing the importance of \ac{xai} research and
correspondingly laid out clear definitions of \ac{xai}-related concepts adapted
from \citet{arrieta2020explainable}. Through our own survey of recent literature
on explainability techniques used in \ac{nlp}, we came across several
interesting studies and drew particular inspiration from
\citet{schwartz2018sopa} who developed the novel \ac{sopa} model. While functioning
well, we found \ac{sopa}'s explainability techniques to be localized and indirect
despite its neural architecture being suited for the globalized and direct
explanations by simplification explainability technique. This inspired our main
objective to propose a modified model \ac{spp}, which could allow for
effective explanations by simplification.

The most significant changes in \ac{spp} include the utility of strict
linear-chain \ac{wfaws} over linear-chain \ac{wfas}, replacement of the \ac{mlp}
in \ac{sopa} with quantized and transparent hidden layers and the introduction of a
globalized and direct explanations by simplification post-hoc explainability
technique to simplify the black-box \ac{spp} model into a transparent \ac{re}
proxy model. With these changes, we proceed to answer our three research
questions detailed in Section \ref{section:rq}.

Regarding our first research question, we observe that \ac{spp}'s best accuracy
range on the \ac{fmtod} data set of 97.6-98.3$\%$ falls into the competitive accuracy
range of 96.6-99-5$\%$ based on other recent studies. In this respect, we
conclude that \ac{spp} offers competitive performance on the \ac{fmtod} English
language intent classification task.

Regarding our second research question, we compare the accuracy scores and
distance metrics between \ac{spp} and \ac{re} proxy model pairs and observe accuracy
differences as low as $\sim$0.1$\%$ and softmax distance norms as small as $\sim$4$\%$ for
medium and large-sized models with $\tau$-thresholds ranging from 0.50-1.00. We
therefore conclude that the explanations by simplification post-hoc
explainability technique is effective on the \ac{fmtod} English language intent
classification task given larger model sizes and $\tau$-thresholds.

Regarding our third and final research question, we identify salient \ac{tauste}
neurons which received disproportionately large relative linear weights in our
best performing small \ac{re} proxy model. Next, we analyze regular expression
samples in the \ac{re} lookup layer from the aforementioned \ac{re} proxy model
corresponding to these salient neurons. Based on an analysis of these sampled
regular expressions, we observe several interesting phenomena such as lexical
segmentation, branching transitions with tokens having similar lexical semantics
and also the presence of USA-centric inductive biases captured from the training
data.

\clearpage

With these answers to our research questions, we addressed our objective of
proposing a modified \ac{spp} model that allows for effective explanations by
simplification. Furthermore, with our competitive SoPa++ model and its simpler
RE proxy; we showed that well-performing black-box models do not have to
significantly compromise performance when undergoing explainability. However, we
did encounter several limitations to the \ac{spp} model while answering our
research questions. In the next chapter, we expound on these limitations and how
they could be addressed in future research.

% LocalWords:  explainability quantized softmax centric

%%% Local Variables: 
%%% mode: latex
%%% TeX-master: "main"
%%% End: 

\chapter{Further work}

\label{chapter:further_work}

In this final chapter, we address the limitations of our thesis and provide
ideas for future research which could address these limitations.

\section{Efficiency}

In this thesis, we showed the effectiveness of simplifying SoPa++ models into RE
proxy models. In the course of our experiments, we realized that while this
process was effective; the simplification and evaluation processes using RE
proxy models tended to be slow. This is largely because we used single-threaded
processes for both simplification and evaluation of RE proxy models; in contrast
to PyTorch's parallelized functionalities on a GPU which made SoPa++ very fast.
To overcome the low efficiency of simplifcation and evaluation using RE proxy
models, we could recommend two approaches. One approach could be to save the RE
lookup layer as a data base and utilize indexed searches to make regular
expression searches and lookups much faster. Another approach could be to
utilize GPU-accelerated regular expression matching algorithms to parallelize
the overall RE lookup layer and its searching functionalities
\citep{wang2011gregex,zu2012gpu,yu2013gpu}.

\section{Explainability}

As mentioned in Section \ref{section:evaluate_explain}, we were only able to
address the technical requirements of explanations by simplification without
necessarily delving into how good the explanations provided are for given target
audiences. This process is subjective and would require a thorough survey with a
given target audience. Conducting such a survey of quality of explanations could
be useful to further evaluate the explainability of SoPa++ and its RE proxy
counterparts.

\section{Discovery and correction of inductive biases}

As shown in Section \ref{section:discussion_regex}, we provided examples of how
a human could manually find and correct inductive biases in the RE lookup layer.
It would be interesting to explore how extensively this process could be
conducted to find and correct inductive biases in the REs inside the RE lookup
layer. Furthermore, it would be interesting if we could also find adversarial
samples in this layer. While correcting the RE lookup layer ``fixes'' problems
on the RE proxy model's side, it does not modify the functionality of the
antecedent SoPa++ model. It would therefore also be interesting to explore how
we could propagate the corrections in the RE lookup layer back into the SoPa++
model's neural network components.

\section{Generalization}

Based on sampled regular expressions shown in Figures
\ref{fig:regex_example_neuron_21}, \ref{fig:regex_example_neuron_27} and
\ref{fig:regex_example_neuron_32}, we can observe certain transitions which
contain many possible words. We can conduct semantic analyses on these highly
utilized transitions and perhaps generalize these to include more tokens which
may even be outside of the initial model vocabulary. For example, in the fourth
RE from the top in Figure \ref{fig:regex_example_neuron_21}, we can observe that
the third transition generally captures numbers formatted as time. We could
generalize this transition to capture any tokens that express the time; which
could lead to generalization on previously unseen data instances. This could
also address issues of unknown tokens altogether. 

\section{Modeling extensions}

Since our current SoPa++ passes binary outputs from the TauSTE layer to the
linear regression layer, we can infer that the classification outputs will be
discretized as are the TauSTE outputs. As a result, we could transform the
linear regression layer during the SoPa++ to RE proxy simplification phase into
a decision tree; which could make the RE proxy model even more transparent given
a small enough decision tree. This process could however be difficult for larger
models with more TauSTE neurons; therefore this process could be more viable for
smaller models.

Another possible extension could be to extend SoPa++'s weighted finite-state
automata to finite-state transducers; which are highly similar but return scored
seqeuences instead of only path scores. In this way, SoPa++ with finite-state
transducers could even be used for sequence-to-sequence tasks; making it viable
to other applications in NLP such as Neural Machine Translation (NMT). Finally,
it would be interesting to see how SoPa++ performs in tasks given longer
sequence lengths since the FMTOD data set generally contains short input
utterances.

%%% Local Variables: 
%%% mode: latex
%%% TeX-master: "main"
%%% End: 

%----------------------------------------------------------------------------------------
%	BIBLIOGRAPHY
%----------------------------------------------------------------------------------------

\printbibliography[heading=bibintoc]

%----------------------------------------------------------------------------------------

\end{document}  

%%% Local Variables:
%%% mode: latex
%%% TeX-master: t
%%% End: